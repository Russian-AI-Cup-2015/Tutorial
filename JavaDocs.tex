% Added by DocsPostProcessor:
\setlength{\parskip}{0.125in}
% \\ Added by DocsPostProcessor.

% Replaced by DocsPostProcessor:
% \documentclass[11pt]{report}
\documentclass[a4paper]{report}
% \\ Replaced by DocsPostProcessor.

\def\bl{\mbox{}\newline\mbox{}\newline{}}

% Added by DocsPostProcessor:
\usepackage{amssymb,latexsym,amsmath,amscd,mathtext,ifthen}
\usepackage[unicode]{hyperref}
\usepackage{listings}
\usepackage{literat}
\usepackage{graphicx}
\usepackage[top = 2.0cm]{geometry}
\usepackage[T1,T2A]{fontenc}
\usepackage[utf8]{inputenc}
\usepackage[english,russian]{babel}

\input glyphtounicode
\pdfgentounicode=1

\setlength{\skip\footins}{0.5cm}
\setlength{\footnotesep}{0.5cm}
% \\ Added by DocsPostProcessor.

\usepackage{ifthen}
\newcommand{\hide}[2]{
\ifthenelse{\equal{#1}{inherited}}%
{}%
{}%
}
\newcommand{\entityintro}[3]{%
  \hbox to \hsize{%
    \vbox{%
      \hbox to .2in{}%
    }%
    {\bf #1}%
    \dotfill\pageref{#2}%
  }
  \makebox[\hsize]{%
    \parbox{.4in}{}%
    \parbox[l]{5in}{%
      \vspace{1mm}\it%
      #3%
      \vspace{1mm}%
    }%
  }%
}
\newcommand{\isep}[0]{%
\setlength{\itemsep}{-.4ex}
}
\newcommand{\sld}[0]{%
\setlength{\topsep}{0em}
\setlength{\partopsep}{0em}
\setlength{\parskip}{0em}
\setlength{\parsep}{-1em}
}
\newcommand{\headref}[3]{%
\ifthenelse{#1 = 1}{%
\addcontentsline{toc}{section}{\hspace{\qquad}\protect\numberline{}{#3}}%
}{}%
\ifthenelse{#1 = 2}{%
\addcontentsline{toc}{subsection}{\hspace{\qquad}\protect\numerline{}{#3}}%
}{}%
\ifthenelse{#1 = 3}{%
\addcontentsline{toc}{subsubsection}{\hspace{\qquad}\protect\numerline{}{#3}}%
}{}%
\label{#3}%
\makebox[\textwidth][l]{#2 #3}%
}%
\newcommand{\membername}[1]{{\it #1}\linebreak}
\newcommand{\divideents}[1]{\vskip -1em\indent\rule{2in}{.5mm}}
\newcommand{\refdefined}[1]{
\expandafter\ifx\csname r@#1\endcsname\relax
\relax\else
{$($ in \ref{#1}, page \pageref{#1}$)$}
\fi}
\newcommand{\startsection}[4]{
\gdef\classname{#2}
\subsection{\label{#3}{\bf {\sc #1} #2}}{
\rule[1em]{\hsize}{4pt}\vskip -1em
\vskip .1in 
#4
}%
}
\newcommand{\startsubsubsection}[2]{
\subsubsection{\sc #1}{%
\rule[1em]{\hsize}{2pt}%
#2}
}
\usepackage{color}

% Replaced by DocsPostProcessor:
% \date{DoNotModifyDateHere}
\date{Ноябрь --- декабрь, 2015}
% \\ Replaced by DocsPostProcessor.


% Removed by DocsPostProcessor:
% \pagestyle{myheadings}
% \\ Removed by DocsPostProcessor.

\addtocontents{toc}{\protect\def\protect\packagename{}}
\addtocontents{toc}{\protect\def\protect\classname{}}
\markboth{\protect\packagename -- \protect\classname}{\protect\packagename -- \protect\classname}
\oddsidemargin 0in
\evensidemargin 0in
% \topmargin -.8in
\chardef\bslash=`\\
\textheight 9.4in
\textwidth 6.5in

% Replaced by DocsPostProcessor:
% \title{DoNotModifyTitleHere}
\title{
\includegraphics{images/raic-2015-logo-302x140.png}\\
\vspace{0.61in}
\textsc{\Huge CodeRacing 2015}\\
\vspace{0.50in}
\textsc{\LARGE Правила}\\
\textsc{\small Версия 1.1.0}\\
\vspace{2.75in}
\includegraphics{images/mail-ru-and-ssu-logo.png}
}
% \\ Replaced by DocsPostProcessor.

\begin{document}
\maketitle
\sloppy
\raggedright
\tableofcontents

% Added by DocsPostProcessor:
\markboth{qqq}{{\footnotesize CodeRacing 2015}}
\input Tutorial.tex
\raggedright
\sloppy
% \\ Added by DocsPostProcessor.

\gdef\packagename{}
\gdef\classname{}
\newpage
\def\packagename{model}
\chapter{\bf Package model}{
\vskip -.25in
\hbox to \hsize{\it Package Contents\hfil Page}
\rule{\hsize}{.7mm}
\vskip .13in
\hbox{\bf Classes}
\entityintro{Bonus}{l0}{Класс, определяющий бонус --- неподвижный полезный объект.}
\entityintro{BonusType}{l1}{Тип бонуса.}
\entityintro{Car}{l2}{Класс, определяющий кодемобиль.}
\entityintro{CarType}{l3}{Тип кодемобиля.}
\entityintro{CircularUnit}{l4}{Базовый класс для определения круглых объектов.}
\entityintro{Direction}{l5}{Направление.}
\entityintro{Game}{l6}{Предоставляет доступ к различным игровым константам.}
\entityintro{Move}{l7}{Стратегия игрока может управлять кодемобилем посредством установки свойств объекта данного класса.}
\entityintro{OilSlick}{l8}{Класс, определяющий лужу мазута.}
\entityintro{Player}{l9}{Содержит данные о текущем состоянии игрока.}
\entityintro{Projectile}{l10}{Класс, определяющий метательный снаряд.}
\entityintro{ProjectileType}{l11}{Тип метательного снаряда.}
\entityintro{RectangularUnit}{l12}{Базовый класс для определения прямоугольных объектов.}
\entityintro{TileType}{l13}{Тип тайла.}
\entityintro{Unit}{l14}{Базовый класс для определения объектов (\textless \textless юнитов\textgreater \textgreater ) на игровом поле.}
\entityintro{World}{l15}{Этот класс описывает игровой мир.}

% Removed by DocsPostProcessor:
% \vskip .1in
% \rule{\hsize}{.7mm}
% \\ Removed by DocsPostProcessor.

\vskip .1in
\newpage
\section{Classes}{
\startsection{Class}{Bonus}{l0}{%
{\small Класс, определяющий бонус --- неподвижный полезный объект. Содержит также все свойства прямоугольного юнита.}
\vskip .1in 
\startsubsubsection{Declaration}{
\fbox{\vbox{
\hbox{\vbox{\small public 
class 
Bonus}}
\noindent\hbox{\vbox{{\bf extends} RectangularUnit}}
}}}

% Removed by DocsPostProcessor:
% \startsubsubsection{Constructors}{
% \vskip -2em
% \begin{itemize}
% \item{\vskip -1.9ex 
% \membername{Bonus}
% {\tt public {\bf Bonus}( {\tt long } {\bf id},
% {\tt double } {\bf mass},
% {\tt double } {\bf x},
% {\tt double } {\bf y},
% {\tt double } {\bf speedX},
% {\tt double } {\bf speedY},
% {\tt double } {\bf angle},
% {\tt double } {\bf angularSpeed},
% {\tt double } {\bf width},
% {\tt double } {\bf height},
% {\tt BonusType } {\bf type} )
% \label{l16}\label{l17}}%end signature
% }%end item
% \end{itemize}
% }
% \\ Removed by DocsPostProcessor.

\startsubsubsection{Methods}{
\vskip -2em
\begin{itemize}
\item{\vskip -1.9ex 
\membername{getType}
{\tt public BonusType {\bf getType}(  )
\label{l18}\label{l19}}%end signature
\begin{itemize}
\sld
\item{{\bf Returns} - 
Возвращает тип бонуса. 
}%end item
\end{itemize}
}%end item
\end{itemize}
}
\hide{inherited}{
\startsubsubsection{Methods inherited from class {\tt RectangularUnit}}{
\par{\small 
\refdefined{l12}\vskip -2em
\begin{itemize}
\item{\vskip -1.9ex 
\membername{getHeight}
{\tt public double {\bf getHeight}(  )
}%end signature
\begin{itemize}
\sld
\item{{\bf Returns} - 
Возвращает высоту объекта. 
}%end item
\end{itemize}
}%end item
\divideents{getWidth}
\item{\vskip -1.9ex 
\membername{getWidth}
{\tt public double {\bf getWidth}(  )
}%end signature
\begin{itemize}
\sld
\item{{\bf Returns} - 
Возвращает ширину объекта. 
}%end item
\end{itemize}
}%end item
\end{itemize}
}}
\startsubsubsection{Methods inherited from class {\tt Unit}}{
\par{\small 
\refdefined{l14}\vskip -2em
\begin{itemize}
\item{\vskip -1.9ex 
\membername{getAngle}
{\tt public final double {\bf getAngle}(  )
}%end signature
\begin{itemize}
\sld
\item{{\bf Returns} - 
Возвращает угол поворота объекта в радианах. Нулевой угол соответствует направлению оси абсцисс.
 Положительные значения соответствуют повороту по часовой стрелке. 
}%end item
\end{itemize}
}%end item
\divideents{getAngleTo}
\item{\vskip -1.9ex 
\membername{getAngleTo}
{\tt public double {\bf getAngleTo}( {\tt double } {\bf x},
{\tt double } {\bf y} )
}%end signature
\begin{itemize}
\sld
\item{
\sld
{\bf Parameters}
\sld\isep
  \begin{itemize}
\sld\isep
   \item{
\sld
{\tt x} - X-координата точки.}
   \item{
\sld
{\tt y} - Y-координата точки.}
  \end{itemize}
}%end item
\item{{\bf Returns} - 
Возвращает ориентированный угол [{\tt -PI}, {\tt PI}] между направлением
 данного объекта и вектором из центра данного объекта к указанной точке. 
}%end item
\end{itemize}
}%end item
\divideents{getAngleTo}
\item{\vskip -1.9ex 
\membername{getAngleTo}
{\tt public double {\bf getAngleTo}( {\tt Unit } {\bf unit} )
}%end signature
\begin{itemize}
\sld
\item{
\sld
{\bf Parameters}
\sld\isep
  \begin{itemize}
\sld\isep
   \item{
\sld
{\tt unit} - Объект, к центру которого необходимо определить угол.}
  \end{itemize}
}%end item
\item{{\bf Returns} - 
Возвращает ориентированный угол [{\tt -PI}, {\tt PI}] между направлением
 данного объекта и вектором из центра данного объекта к центру указанного объекта. 
}%end item
\end{itemize}
}%end item
\divideents{getAngularSpeed}
\item{\vskip -1.9ex 
\membername{getAngularSpeed}
{\tt public double {\bf getAngularSpeed}(  )
}%end signature
\begin{itemize}
\sld
\item{{\bf Returns} - 
Возвращает скорость вращения объекта.
 Положительные значения соответствуют вращению по часовой стрелке. 
}%end item
\end{itemize}
}%end item
\divideents{getDistanceTo}
\item{\vskip -1.9ex 
\membername{getDistanceTo}
{\tt public double {\bf getDistanceTo}( {\tt double } {\bf x},
{\tt double } {\bf y} )
}%end signature
\begin{itemize}
\sld
\item{
\sld
{\bf Parameters}
\sld\isep
  \begin{itemize}
\sld\isep
   \item{
\sld
{\tt x} - X-координата точки.}
   \item{
\sld
{\tt y} - Y-координата точки.}
  \end{itemize}
}%end item
\item{{\bf Returns} - 
Возвращает расстояние до точки от центра данного объекта. 
}%end item
\end{itemize}
}%end item
\divideents{getDistanceTo}
\item{\vskip -1.9ex 
\membername{getDistanceTo}
{\tt public double {\bf getDistanceTo}( {\tt Unit } {\bf unit} )
}%end signature
\begin{itemize}
\sld
\item{
\sld
{\bf Parameters}
\sld\isep
  \begin{itemize}
\sld\isep
   \item{
\sld
{\tt unit} - Объект, до центра которого необходимо определить расстояние.}
  \end{itemize}
}%end item
\item{{\bf Returns} - 
Возвращает расстояние от центра данного объекта до центра указанного объекта. 
}%end item
\end{itemize}
}%end item
\divideents{getId}
\item{\vskip -1.9ex 
\membername{getId}
{\tt public long {\bf getId}(  )
}%end signature
\begin{itemize}
\sld
\item{{\bf Returns} - 
Возвращает уникальный идентификатор объекта. 
}%end item
\end{itemize}
}%end item
\divideents{getMass}
\item{\vskip -1.9ex 
\membername{getMass}
{\tt public double {\bf getMass}(  )
}%end signature
\begin{itemize}
\sld
\item{{\bf Returns} - 
Возвращает массу объекта в единицах массы. 
}%end item
\end{itemize}
}%end item
\divideents{getSpeedX}
\item{\vskip -1.9ex 
\membername{getSpeedX}
{\tt public final double {\bf getSpeedX}(  )
}%end signature
\begin{itemize}
\sld
\item{{\bf Returns} - 
Возвращает X-составляющую скорости объекта. Ось абсцисс направлена слева направо. 
}%end item
\end{itemize}
}%end item
\divideents{getSpeedY}
\item{\vskip -1.9ex 
\membername{getSpeedY}
{\tt public final double {\bf getSpeedY}(  )
}%end signature
\begin{itemize}
\sld
\item{{\bf Returns} - 
Возвращает Y-составляющую скорости объекта. Ось ординат направлена сверху вниз. 
}%end item
\end{itemize}
}%end item
\divideents{getX}
\item{\vskip -1.9ex 
\membername{getX}
{\tt public final double {\bf getX}(  )
}%end signature
\begin{itemize}
\sld
\item{{\bf Returns} - 
Возвращает X-координату центра объекта. Ось абсцисс направлена слева направо. 
}%end item
\end{itemize}
}%end item
\divideents{getY}
\item{\vskip -1.9ex 
\membername{getY}
{\tt public final double {\bf getY}(  )
}%end signature
\begin{itemize}
\sld
\item{{\bf Returns} - 
Возвращает Y-координату центра объекта. Ось ординат направлена сверху вниз. 
}%end item
\end{itemize}
}%end item
\end{itemize}
}}
}
}
\startsection{Class}{BonusType}{l1}{%
{\small Тип бонуса.}
\vskip .1in 
\startsubsubsection{Declaration}{
\fbox{\vbox{
\hbox{\vbox{\small public final 
class 
BonusType}}
\noindent\hbox{\vbox{{\bf extends} Enum}}
}}}
\startsubsubsection{Fields}{
\begin{itemize}
\item{
public static final BonusType REPAIR\_KIT\begin{itemize}\item{\vskip -.9ex Ремкомплект. Полностью устраняет все повреждения кодемобиля при подборе.}\end{itemize}
}
\item{
public static final BonusType AMMO\_CRATE\begin{itemize}\item{\vskip -.9ex Ящик с метательными снарядами. При подборе пополняет запас снарядов кодемобиля на единицу. Тип снарядов
 соответствует типу кодемобиля.}\end{itemize}
}
\item{
public static final BonusType NITRO\_BOOST\begin{itemize}\item{\vskip -.9ex Топливо для системы закиси азота. При использовании мгновенно устанавливает мощность двигателя кодемобиля равной
 значению {\tt game.nitroEnginePowerFactor} и не даёт изменять её в течение {\tt game.useNitroCooldownTicks}
 тиков.}\end{itemize}
}
\item{
public static final BonusType OIL\_CANISTER\begin{itemize}\item{\vskip -.9ex Канистра с мазутом. При использовании на {\tt game.oilSlickLifetime} тиков создаёт позади кодемобиля скользкую
 круглую область. Радиус области равен {\tt game.oilSlickRadius}, центр находится на продольной оси кодемобиля,
 а расстояние между ближайшими точками области и кодемобиля составляет {\tt game.oilSlickInitialRange}.}\end{itemize}
}
\item{
public static final BonusType PURE\_SCORE\begin{itemize}\item{\vskip -.9ex Баллы в чистом виде. При подборе данного бонуса игроку мгновенно начисляется {\tt game.pureScoreAmount} баллов.}\end{itemize}
}
\end{itemize}
}
\startsubsubsection{Methods}{
\vskip -2em
\begin{itemize}
\item{\vskip -1.9ex 
\membername{valueOf}
{\tt public static BonusType {\bf valueOf}( {\tt String } {\bf name} )
\label{l20}\label{l21}}%end signature
}%end item
\divideents{values}
\item{\vskip -1.9ex 
\membername{values}
{\tt public static BonusType[] {\bf values}(  )
\label{l22}\label{l23}}%end signature
}%end item
\end{itemize}
}
\hide{inherited}{
\startsubsubsection{Methods inherited from class {\tt Enum}}{
\par{\small 
\refdefined{l24}\vskip -2em
\begin{itemize}
\item{\vskip -1.9ex 
\membername{clone}
{\tt protected final Object {\bf clone}(  )
}%end signature
}%end item
\divideents{compareTo}
\item{\vskip -1.9ex 
\membername{compareTo}
{\tt public final int {\bf compareTo}( {\tt Enum } {\bf arg0} )
}%end signature
}%end item
\divideents{equals}
\item{\vskip -1.9ex 
\membername{equals}
{\tt public final boolean {\bf equals}( {\tt Object } {\bf arg0} )
}%end signature
}%end item
\divideents{finalize}
\item{\vskip -1.9ex 
\membername{finalize}
{\tt protected final void {\bf finalize}(  )
}%end signature
}%end item
\divideents{getDeclaringClass}
\item{\vskip -1.9ex 
\membername{getDeclaringClass}
{\tt public final Class {\bf getDeclaringClass}(  )
}%end signature
}%end item
\divideents{hashCode}
\item{\vskip -1.9ex 
\membername{hashCode}
{\tt public final int {\bf hashCode}(  )
}%end signature
}%end item
\divideents{name}
\item{\vskip -1.9ex 
\membername{name}
{\tt public final String {\bf name}(  )
}%end signature
}%end item
\divideents{ordinal}
\item{\vskip -1.9ex 
\membername{ordinal}
{\tt public final int {\bf ordinal}(  )
}%end signature
}%end item
\divideents{toString}
\item{\vskip -1.9ex 
\membername{toString}
{\tt public String {\bf toString}(  )
}%end signature
}%end item
\divideents{valueOf}
\item{\vskip -1.9ex 
\membername{valueOf}
{\tt public static Enum {\bf valueOf}( {\tt Class } {\bf arg0},
{\tt String } {\bf arg1} )
}%end signature
}%end item
\end{itemize}
}}
}
}
\startsection{Class}{Car}{l2}{%
{\small Класс, определяющий кодемобиль. Содержит также все свойства прямоугольного юнита.}
\vskip .1in 
\startsubsubsection{Declaration}{
\fbox{\vbox{
\hbox{\vbox{\small public 
class 
Car}}
\noindent\hbox{\vbox{{\bf extends} RectangularUnit}}
}}}

% Removed by DocsPostProcessor:
% \startsubsubsection{Constructors}{
% \vskip -2em
% \begin{itemize}
% \item{\vskip -1.9ex 
% \membername{Car}
% {\tt public {\bf Car}( {\tt long } {\bf id},
% {\tt double } {\bf mass},
% {\tt double } {\bf x},
% {\tt double } {\bf y},
% {\tt double } {\bf speedX},
% {\tt double } {\bf speedY},
% {\tt double } {\bf angle},
% {\tt double } {\bf angularSpeed},
% {\tt double } {\bf width},
% {\tt double } {\bf height},
% {\tt long } {\bf playerId},
% {\tt int } {\bf teammateIndex},
% {\tt boolean } {\bf teammate},
% {\tt CarType } {\bf type},
% {\tt int } {\bf projectileCount},
% {\tt int } {\bf nitroChargeCount},
% {\tt int } {\bf oilCanisterCount},
% {\tt int } {\bf remainingProjectileCooldownTicks},
% {\tt int } {\bf remainingNitroCooldownTicks},
% {\tt int } {\bf remainingOilCooldownTicks},
% {\tt int } {\bf remainingNitroTicks},
% {\tt int } {\bf remainingOiledTicks},
% {\tt double } {\bf durability},
% {\tt double } {\bf enginePower},
% {\tt double } {\bf wheelTurn},
% {\tt int } {\bf nextWaypointIndex},
% {\tt int } {\bf nextWaypointX},
% {\tt int } {\bf nextWaypointY},
% {\tt boolean } {\bf finishedTrack} )
% \label{l25}\label{l26}}%end signature
% }%end item
% \end{itemize}
% }
% \\ Removed by DocsPostProcessor.

\startsubsubsection{Methods}{
\vskip -2em
\begin{itemize}
\item{\vskip -1.9ex 
\membername{getDurability}
{\tt public double {\bf getDurability}(  )
\label{l27}\label{l28}}%end signature
\begin{itemize}
\sld
\item{{\bf Returns} - 
Возвращает текущую прочность кодемобиля в интервале [{\tt 0.0}, {\tt 1.0}]. 
}%end item
\end{itemize}
}%end item
\divideents{getEnginePower}
\item{\vskip -1.9ex 
\membername{getEnginePower}
{\tt public double {\bf getEnginePower}(  )
\label{l29}\label{l30}}%end signature
\begin{itemize}
\sld
\item{{\bf Returns} - 
Возвращает относительную мощность двигателя кодемобиля. Значение находится в интервале
 [{\tt -1.0}, {\tt 1.0}] кроме случаев, когда кодемобиль использует ускорение \textless \textless нитро\textgreater \textgreater . 
}%end item
\end{itemize}
}%end item
\divideents{getNextWaypointIndex}
\item{\vskip -1.9ex 
\membername{getNextWaypointIndex}
{\tt public int {\bf getNextWaypointIndex}(  )
\label{l31}\label{l32}}%end signature
\begin{itemize}
\sld
\item{{\bf Returns} - 
Возвращает индекс следующего ключевого тайла в массиве {\tt world.waypoints}. 
}%end item
\end{itemize}
}%end item
\divideents{getNextWaypointX}
\item{\vskip -1.9ex 
\membername{getNextWaypointX}
{\tt public int {\bf getNextWaypointX}(  )
\label{l33}\label{l34}}%end signature
\begin{itemize}
\sld
\item{{\bf Returns} - 
Возвращает компоненту X позиции следующего ключевого тайла.
 Конвертировать позицию в точные координаты можно, используя значение {\tt game.trackTileSize}. 
}%end item
\end{itemize}
}%end item
\divideents{getNextWaypointY}
\item{\vskip -1.9ex 
\membername{getNextWaypointY}
{\tt public int {\bf getNextWaypointY}(  )
\label{l35}\label{l36}}%end signature
\begin{itemize}
\sld
\item{{\bf Returns} - 
Возвращает компоненту Y позиции следующего ключевого тайла.
 Конвертировать позицию в точные координаты можно, используя значение {\tt game.trackTileSize}. 
}%end item
\end{itemize}
}%end item
\divideents{getNitroChargeCount}
\item{\vskip -1.9ex 
\membername{getNitroChargeCount}
{\tt public int {\bf getNitroChargeCount}(  )
\label{l37}\label{l38}}%end signature
\begin{itemize}
\sld
\item{{\bf Returns} - 
Возвращает количество зарядов для системы закиси азота. 
}%end item
\end{itemize}
}%end item
\divideents{getOilCanisterCount}
\item{\vskip -1.9ex 
\membername{getOilCanisterCount}
{\tt public int {\bf getOilCanisterCount}(  )
\label{l39}\label{l40}}%end signature
\begin{itemize}
\sld
\item{{\bf Returns} - 
Возвращает количество канистр с мазутом. 
}%end item
\end{itemize}
}%end item
\divideents{getPlayerId}
\item{\vskip -1.9ex 
\membername{getPlayerId}
{\tt public long {\bf getPlayerId}(  )
\label{l41}\label{l42}}%end signature
\begin{itemize}
\sld
\item{{\bf Returns} - 
Возвращает идентификатор игрока, которому принадлежит кодемобиль. 
}%end item
\end{itemize}
}%end item
\divideents{getProjectileCount}
\item{\vskip -1.9ex 
\membername{getProjectileCount}
{\tt public int {\bf getProjectileCount}(  )
\label{l43}\label{l44}}%end signature
\begin{itemize}
\sld
\item{{\bf Returns} - 
Возвращает количество метательных снарядов. 
}%end item
\end{itemize}
}%end item
\divideents{getRemainingNitroCooldownTicks}
\item{\vskip -1.9ex 
\membername{getRemainingNitroCooldownTicks}
{\tt public int {\bf getRemainingNitroCooldownTicks}(  )
\label{l45}\label{l46}}%end signature
\begin{itemize}
\sld
\item{{\bf Returns} - 
Возвращает количество тиков, по прошествии которого кодемобиль может использовать очередной заряд системы
 закиси азота, или {\tt 0}, если кодемобиль может совершить данное действие в текущий тик. 
}%end item
\end{itemize}
}%end item
\divideents{getRemainingNitroTicks}
\item{\vskip -1.9ex 
\membername{getRemainingNitroTicks}
{\tt public int {\bf getRemainingNitroTicks}(  )
\label{l47}\label{l48}}%end signature
\begin{itemize}
\sld
\item{{\bf Returns} - 
Возвращает количество оставшихся тиков действия системы закиси азота. 
}%end item
\end{itemize}
}%end item
\divideents{getRemainingOilCooldownTicks}
\item{\vskip -1.9ex 
\membername{getRemainingOilCooldownTicks}
{\tt public int {\bf getRemainingOilCooldownTicks}(  )
\label{l49}\label{l50}}%end signature
\begin{itemize}
\sld
\item{{\bf Returns} - 
Возвращает количество тиков, по прошествии которого кодемобиль может разлить очередную лужу мазута,
 или {\tt 0}, если кодемобиль может совершить данное действие в текущий тик. 
}%end item
\end{itemize}
}%end item
\divideents{getRemainingOiledTicks}
\item{\vskip -1.9ex 
\membername{getRemainingOiledTicks}
{\tt public int {\bf getRemainingOiledTicks}(  )
\label{l51}\label{l52}}%end signature
\begin{itemize}
\sld
\item{{\bf Returns} - 
Возвращает количество тиков, оставшихся до полного высыхания кодемобиля, попавшего в лужу мазута. 
}%end item
\end{itemize}
}%end item
\divideents{getRemainingProjectileCooldownTicks}
\item{\vskip -1.9ex 
\membername{getRemainingProjectileCooldownTicks}
{\tt public int {\bf getRemainingProjectileCooldownTicks}(  )
\label{l53}\label{l54}}%end signature
\begin{itemize}
\sld
\item{{\bf Returns} - 
Возвращает количество тиков, по прошествии которого кодемобиль может запустить очередной снаряд,
 или {\tt 0}, если кодемобиль может совершить данное действие в текущий тик. 
}%end item
\end{itemize}
}%end item
\divideents{getTeammateIndex}
\item{\vskip -1.9ex 
\membername{getTeammateIndex}
{\tt public int {\bf getTeammateIndex}(  )
\label{l55}\label{l56}}%end signature
\begin{itemize}
\sld
\item{{\bf Returns} - 
Возвращает 0-индексированный номер кодемобиля среди юнитов одного игрока. 
}%end item
\end{itemize}
}%end item
\divideents{getType}
\item{\vskip -1.9ex 
\membername{getType}
{\tt public CarType {\bf getType}(  )
\label{l57}\label{l58}}%end signature
\begin{itemize}
\sld
\item{{\bf Returns} - 
Возвращает тип кодемобиля. 
}%end item
\end{itemize}
}%end item
\divideents{getWheelTurn}
\item{\vskip -1.9ex 
\membername{getWheelTurn}
{\tt public double {\bf getWheelTurn}(  )
\label{l59}\label{l60}}%end signature
\begin{itemize}
\sld
\item{{\bf Returns} - 
Возвращает относительный угол поворота колёс (или руля, что эквивалентно) кодемобиля в интервале
 [{\tt -1.0}, {\tt 1.0}]. 
}%end item
\end{itemize}
}%end item
\divideents{isFinishedTrack}
\item{\vskip -1.9ex 
\membername{isFinishedTrack}
{\tt public boolean {\bf isFinishedTrack}(  )
\label{l61}\label{l62}}%end signature
\begin{itemize}
\sld
\item{{\bf Returns} - 
Возвращает {\tt true}, если и только если данный кодемобиль финишировал. Финишировавший кодемобиль
 перестаёт управляться игроком, а также участвовать в столкновениях с другими юнитами. 
}%end item
\end{itemize}
}%end item
\divideents{isTeammate}
\item{\vskip -1.9ex 
\membername{isTeammate}
{\tt public boolean {\bf isTeammate}(  )
\label{l63}\label{l64}}%end signature
\begin{itemize}
\sld
\item{{\bf Returns} - 
Возвращает {\tt true}, если и только если данный кодемобиль принадлежит вам. 
}%end item
\end{itemize}
}%end item
\end{itemize}
}
\hide{inherited}{
\startsubsubsection{Methods inherited from class {\tt RectangularUnit}}{
\par{\small 
\refdefined{l12}\vskip -2em
\begin{itemize}
\item{\vskip -1.9ex 
\membername{getHeight}
{\tt public double {\bf getHeight}(  )
}%end signature
\begin{itemize}
\sld
\item{{\bf Returns} - 
Возвращает высоту объекта. 
}%end item
\end{itemize}
}%end item
\divideents{getWidth}
\item{\vskip -1.9ex 
\membername{getWidth}
{\tt public double {\bf getWidth}(  )
}%end signature
\begin{itemize}
\sld
\item{{\bf Returns} - 
Возвращает ширину объекта. 
}%end item
\end{itemize}
}%end item
\end{itemize}
}}
\startsubsubsection{Methods inherited from class {\tt Unit}}{
\par{\small 
\refdefined{l14}\vskip -2em
\begin{itemize}
\item{\vskip -1.9ex 
\membername{getAngle}
{\tt public final double {\bf getAngle}(  )
}%end signature
\begin{itemize}
\sld
\item{{\bf Returns} - 
Возвращает угол поворота объекта в радианах. Нулевой угол соответствует направлению оси абсцисс.
 Положительные значения соответствуют повороту по часовой стрелке. 
}%end item
\end{itemize}
}%end item
\divideents{getAngleTo}
\item{\vskip -1.9ex 
\membername{getAngleTo}
{\tt public double {\bf getAngleTo}( {\tt double } {\bf x},
{\tt double } {\bf y} )
}%end signature
\begin{itemize}
\sld
\item{
\sld
{\bf Parameters}
\sld\isep
  \begin{itemize}
\sld\isep
   \item{
\sld
{\tt x} - X-координата точки.}
   \item{
\sld
{\tt y} - Y-координата точки.}
  \end{itemize}
}%end item
\item{{\bf Returns} - 
Возвращает ориентированный угол [{\tt -PI}, {\tt PI}] между направлением
 данного объекта и вектором из центра данного объекта к указанной точке. 
}%end item
\end{itemize}
}%end item
\divideents{getAngleTo}
\item{\vskip -1.9ex 
\membername{getAngleTo}
{\tt public double {\bf getAngleTo}( {\tt Unit } {\bf unit} )
}%end signature
\begin{itemize}
\sld
\item{
\sld
{\bf Parameters}
\sld\isep
  \begin{itemize}
\sld\isep
   \item{
\sld
{\tt unit} - Объект, к центру которого необходимо определить угол.}
  \end{itemize}
}%end item
\item{{\bf Returns} - 
Возвращает ориентированный угол [{\tt -PI}, {\tt PI}] между направлением
 данного объекта и вектором из центра данного объекта к центру указанного объекта. 
}%end item
\end{itemize}
}%end item
\divideents{getAngularSpeed}
\item{\vskip -1.9ex 
\membername{getAngularSpeed}
{\tt public double {\bf getAngularSpeed}(  )
}%end signature
\begin{itemize}
\sld
\item{{\bf Returns} - 
Возвращает скорость вращения объекта.
 Положительные значения соответствуют вращению по часовой стрелке. 
}%end item
\end{itemize}
}%end item
\divideents{getDistanceTo}
\item{\vskip -1.9ex 
\membername{getDistanceTo}
{\tt public double {\bf getDistanceTo}( {\tt double } {\bf x},
{\tt double } {\bf y} )
}%end signature
\begin{itemize}
\sld
\item{
\sld
{\bf Parameters}
\sld\isep
  \begin{itemize}
\sld\isep
   \item{
\sld
{\tt x} - X-координата точки.}
   \item{
\sld
{\tt y} - Y-координата точки.}
  \end{itemize}
}%end item
\item{{\bf Returns} - 
Возвращает расстояние до точки от центра данного объекта. 
}%end item
\end{itemize}
}%end item
\divideents{getDistanceTo}
\item{\vskip -1.9ex 
\membername{getDistanceTo}
{\tt public double {\bf getDistanceTo}( {\tt Unit } {\bf unit} )
}%end signature
\begin{itemize}
\sld
\item{
\sld
{\bf Parameters}
\sld\isep
  \begin{itemize}
\sld\isep
   \item{
\sld
{\tt unit} - Объект, до центра которого необходимо определить расстояние.}
  \end{itemize}
}%end item
\item{{\bf Returns} - 
Возвращает расстояние от центра данного объекта до центра указанного объекта. 
}%end item
\end{itemize}
}%end item
\divideents{getId}
\item{\vskip -1.9ex 
\membername{getId}
{\tt public long {\bf getId}(  )
}%end signature
\begin{itemize}
\sld
\item{{\bf Returns} - 
Возвращает уникальный идентификатор объекта. 
}%end item
\end{itemize}
}%end item
\divideents{getMass}
\item{\vskip -1.9ex 
\membername{getMass}
{\tt public double {\bf getMass}(  )
}%end signature
\begin{itemize}
\sld
\item{{\bf Returns} - 
Возвращает массу объекта в единицах массы. 
}%end item
\end{itemize}
}%end item
\divideents{getSpeedX}
\item{\vskip -1.9ex 
\membername{getSpeedX}
{\tt public final double {\bf getSpeedX}(  )
}%end signature
\begin{itemize}
\sld
\item{{\bf Returns} - 
Возвращает X-составляющую скорости объекта. Ось абсцисс направлена слева направо. 
}%end item
\end{itemize}
}%end item
\divideents{getSpeedY}
\item{\vskip -1.9ex 
\membername{getSpeedY}
{\tt public final double {\bf getSpeedY}(  )
}%end signature
\begin{itemize}
\sld
\item{{\bf Returns} - 
Возвращает Y-составляющую скорости объекта. Ось ординат направлена сверху вниз. 
}%end item
\end{itemize}
}%end item
\divideents{getX}
\item{\vskip -1.9ex 
\membername{getX}
{\tt public final double {\bf getX}(  )
}%end signature
\begin{itemize}
\sld
\item{{\bf Returns} - 
Возвращает X-координату центра объекта. Ось абсцисс направлена слева направо. 
}%end item
\end{itemize}
}%end item
\divideents{getY}
\item{\vskip -1.9ex 
\membername{getY}
{\tt public final double {\bf getY}(  )
}%end signature
\begin{itemize}
\sld
\item{{\bf Returns} - 
Возвращает Y-координату центра объекта. Ось ординат направлена сверху вниз. 
}%end item
\end{itemize}
}%end item
\end{itemize}
}}
}
}
\startsection{Class}{CarType}{l3}{%
{\small Тип кодемобиля.
 
% Removed by DocsPostProcessor:
% \textless p$/$\textgreater
% \\ Removed by DocsPostProcessor.
 
 В Раунде 1 чемпионата стратегия игрока управляет одним кодемобилем типа {\tt BUGGY}.
 В гонке участвуют {\tt 4} игрока.
 
% Removed by DocsPostProcessor:
% \textless p$/$\textgreater
% \\ Removed by DocsPostProcessor.
 
 В Раунде 2 чемпионата стратегия игрока управляет одним кодемобилем типа {\tt JEEP}.
 В гонке участвуют {\tt 4} игрока.
 
% Removed by DocsPostProcessor:
% \textless p$/$\textgreater
% \\ Removed by DocsPostProcessor.
 
 В Финале в распоряжении стратегии игрока находится по одному кодемобилю каждого типа.
 В гонке участвуют {\tt 2} игрока.}
\vskip .1in 
\startsubsubsection{Declaration}{
\fbox{\vbox{
\hbox{\vbox{\small public final 
class 
CarType}}
\noindent\hbox{\vbox{{\bf extends} Enum}}
}}}
\startsubsubsection{Fields}{
\begin{itemize}
\item{
public static final CarType BUGGY\begin{itemize}\item{\vskip -.9ex Багги. Стреляет тремя небольшими шайбами, расходящимися под небольшим углом. Немного легче джипа.}\end{itemize}
}
\item{
public static final CarType JEEP\begin{itemize}\item{\vskip -.9ex Джип. Стреляет большими массивными шинами, отскакивающими от машин и границ трассы. Немного тяжелее багги.}\end{itemize}
}
\end{itemize}
}
\startsubsubsection{Methods}{
\vskip -2em
\begin{itemize}
\item{\vskip -1.9ex 
\membername{valueOf}
{\tt public static CarType {\bf valueOf}( {\tt String } {\bf name} )
\label{l65}\label{l66}}%end signature
}%end item
\divideents{values}
\item{\vskip -1.9ex 
\membername{values}
{\tt public static CarType[] {\bf values}(  )
\label{l67}\label{l68}}%end signature
}%end item
\end{itemize}
}
\hide{inherited}{
\startsubsubsection{Methods inherited from class {\tt Enum}}{
\par{\small 
\refdefined{l24}\vskip -2em
\begin{itemize}
\item{\vskip -1.9ex 
\membername{clone}
{\tt protected final Object {\bf clone}(  )
}%end signature
}%end item
\divideents{compareTo}
\item{\vskip -1.9ex 
\membername{compareTo}
{\tt public final int {\bf compareTo}( {\tt Enum } {\bf arg0} )
}%end signature
}%end item
\divideents{equals}
\item{\vskip -1.9ex 
\membername{equals}
{\tt public final boolean {\bf equals}( {\tt Object } {\bf arg0} )
}%end signature
}%end item
\divideents{finalize}
\item{\vskip -1.9ex 
\membername{finalize}
{\tt protected final void {\bf finalize}(  )
}%end signature
}%end item
\divideents{getDeclaringClass}
\item{\vskip -1.9ex 
\membername{getDeclaringClass}
{\tt public final Class {\bf getDeclaringClass}(  )
}%end signature
}%end item
\divideents{hashCode}
\item{\vskip -1.9ex 
\membername{hashCode}
{\tt public final int {\bf hashCode}(  )
}%end signature
}%end item
\divideents{name}
\item{\vskip -1.9ex 
\membername{name}
{\tt public final String {\bf name}(  )
}%end signature
}%end item
\divideents{ordinal}
\item{\vskip -1.9ex 
\membername{ordinal}
{\tt public final int {\bf ordinal}(  )
}%end signature
}%end item
\divideents{toString}
\item{\vskip -1.9ex 
\membername{toString}
{\tt public String {\bf toString}(  )
}%end signature
}%end item
\divideents{valueOf}
\item{\vskip -1.9ex 
\membername{valueOf}
{\tt public static Enum {\bf valueOf}( {\tt Class } {\bf arg0},
{\tt String } {\bf arg1} )
}%end signature
}%end item
\end{itemize}
}}
}
}
\startsection{Class}{CircularUnit}{l4}{%
{\small Базовый класс для определения круглых объектов. Содержит также все свойства юнита.}
\vskip .1in 
\startsubsubsection{Declaration}{
\fbox{\vbox{
\hbox{\vbox{\small public abstract 
class 
CircularUnit}}
\noindent\hbox{\vbox{{\bf extends} Unit}}
}}}

% Removed by DocsPostProcessor:
% \startsubsubsection{Constructors}{
% \vskip -2em
% \begin{itemize}
% \item{\vskip -1.9ex 
% \membername{CircularUnit}
% {\tt protected {\bf CircularUnit}( {\tt long } {\bf id},
% {\tt double } {\bf mass},
% {\tt double } {\bf x},
% {\tt double } {\bf y},
% {\tt double } {\bf speedX},
% {\tt double } {\bf speedY},
% {\tt double } {\bf angle},
% {\tt double } {\bf angularSpeed},
% {\tt double } {\bf radius} )
% \label{l69}\label{l70}}%end signature
% }%end item
% \end{itemize}
% }
% \\ Removed by DocsPostProcessor.

\startsubsubsection{Methods}{
\vskip -2em
\begin{itemize}
\item{\vskip -1.9ex 
\membername{getRadius}
{\tt public double {\bf getRadius}(  )
\label{l71}\label{l72}}%end signature
\begin{itemize}
\sld
\item{{\bf Returns} - 
Возвращает радиус объекта. 
}%end item
\end{itemize}
}%end item
\end{itemize}
}
\hide{inherited}{
\startsubsubsection{Methods inherited from class {\tt Unit}}{
\par{\small 
\refdefined{l14}\vskip -2em
\begin{itemize}
\item{\vskip -1.9ex 
\membername{getAngle}
{\tt public final double {\bf getAngle}(  )
}%end signature
\begin{itemize}
\sld
\item{{\bf Returns} - 
Возвращает угол поворота объекта в радианах. Нулевой угол соответствует направлению оси абсцисс.
 Положительные значения соответствуют повороту по часовой стрелке. 
}%end item
\end{itemize}
}%end item
\divideents{getAngleTo}
\item{\vskip -1.9ex 
\membername{getAngleTo}
{\tt public double {\bf getAngleTo}( {\tt double } {\bf x},
{\tt double } {\bf y} )
}%end signature
\begin{itemize}
\sld
\item{
\sld
{\bf Parameters}
\sld\isep
  \begin{itemize}
\sld\isep
   \item{
\sld
{\tt x} - X-координата точки.}
   \item{
\sld
{\tt y} - Y-координата точки.}
  \end{itemize}
}%end item
\item{{\bf Returns} - 
Возвращает ориентированный угол [{\tt -PI}, {\tt PI}] между направлением
 данного объекта и вектором из центра данного объекта к указанной точке. 
}%end item
\end{itemize}
}%end item
\divideents{getAngleTo}
\item{\vskip -1.9ex 
\membername{getAngleTo}
{\tt public double {\bf getAngleTo}( {\tt Unit } {\bf unit} )
}%end signature
\begin{itemize}
\sld
\item{
\sld
{\bf Parameters}
\sld\isep
  \begin{itemize}
\sld\isep
   \item{
\sld
{\tt unit} - Объект, к центру которого необходимо определить угол.}
  \end{itemize}
}%end item
\item{{\bf Returns} - 
Возвращает ориентированный угол [{\tt -PI}, {\tt PI}] между направлением
 данного объекта и вектором из центра данного объекта к центру указанного объекта. 
}%end item
\end{itemize}
}%end item
\divideents{getAngularSpeed}
\item{\vskip -1.9ex 
\membername{getAngularSpeed}
{\tt public double {\bf getAngularSpeed}(  )
}%end signature
\begin{itemize}
\sld
\item{{\bf Returns} - 
Возвращает скорость вращения объекта.
 Положительные значения соответствуют вращению по часовой стрелке. 
}%end item
\end{itemize}
}%end item
\divideents{getDistanceTo}
\item{\vskip -1.9ex 
\membername{getDistanceTo}
{\tt public double {\bf getDistanceTo}( {\tt double } {\bf x},
{\tt double } {\bf y} )
}%end signature
\begin{itemize}
\sld
\item{
\sld
{\bf Parameters}
\sld\isep
  \begin{itemize}
\sld\isep
   \item{
\sld
{\tt x} - X-координата точки.}
   \item{
\sld
{\tt y} - Y-координата точки.}
  \end{itemize}
}%end item
\item{{\bf Returns} - 
Возвращает расстояние до точки от центра данного объекта. 
}%end item
\end{itemize}
}%end item
\divideents{getDistanceTo}
\item{\vskip -1.9ex 
\membername{getDistanceTo}
{\tt public double {\bf getDistanceTo}( {\tt Unit } {\bf unit} )
}%end signature
\begin{itemize}
\sld
\item{
\sld
{\bf Parameters}
\sld\isep
  \begin{itemize}
\sld\isep
   \item{
\sld
{\tt unit} - Объект, до центра которого необходимо определить расстояние.}
  \end{itemize}
}%end item
\item{{\bf Returns} - 
Возвращает расстояние от центра данного объекта до центра указанного объекта. 
}%end item
\end{itemize}
}%end item
\divideents{getId}
\item{\vskip -1.9ex 
\membername{getId}
{\tt public long {\bf getId}(  )
}%end signature
\begin{itemize}
\sld
\item{{\bf Returns} - 
Возвращает уникальный идентификатор объекта. 
}%end item
\end{itemize}
}%end item
\divideents{getMass}
\item{\vskip -1.9ex 
\membername{getMass}
{\tt public double {\bf getMass}(  )
}%end signature
\begin{itemize}
\sld
\item{{\bf Returns} - 
Возвращает массу объекта в единицах массы. 
}%end item
\end{itemize}
}%end item
\divideents{getSpeedX}
\item{\vskip -1.9ex 
\membername{getSpeedX}
{\tt public final double {\bf getSpeedX}(  )
}%end signature
\begin{itemize}
\sld
\item{{\bf Returns} - 
Возвращает X-составляющую скорости объекта. Ось абсцисс направлена слева направо. 
}%end item
\end{itemize}
}%end item
\divideents{getSpeedY}
\item{\vskip -1.9ex 
\membername{getSpeedY}
{\tt public final double {\bf getSpeedY}(  )
}%end signature
\begin{itemize}
\sld
\item{{\bf Returns} - 
Возвращает Y-составляющую скорости объекта. Ось ординат направлена сверху вниз. 
}%end item
\end{itemize}
}%end item
\divideents{getX}
\item{\vskip -1.9ex 
\membername{getX}
{\tt public final double {\bf getX}(  )
}%end signature
\begin{itemize}
\sld
\item{{\bf Returns} - 
Возвращает X-координату центра объекта. Ось абсцисс направлена слева направо. 
}%end item
\end{itemize}
}%end item
\divideents{getY}
\item{\vskip -1.9ex 
\membername{getY}
{\tt public final double {\bf getY}(  )
}%end signature
\begin{itemize}
\sld
\item{{\bf Returns} - 
Возвращает Y-координату центра объекта. Ось ординат направлена сверху вниз. 
}%end item
\end{itemize}
}%end item
\end{itemize}
}}
}
}
\startsection{Class}{Direction}{l5}{%
{\small Направление.}
\vskip .1in 
\startsubsubsection{Declaration}{
\fbox{\vbox{
\hbox{\vbox{\small public final 
class 
Direction}}
\noindent\hbox{\vbox{{\bf extends} Enum}}
}}}
\startsubsubsection{Fields}{
\begin{itemize}
\item{
public static final Direction LEFT\begin{itemize}\item{\vskip -.9ex Налево$/$слева.}\end{itemize}
}
\item{
public static final Direction RIGHT\begin{itemize}\item{\vskip -.9ex Направо$/$справа.}\end{itemize}
}
\item{
public static final Direction UP\begin{itemize}\item{\vskip -.9ex Вверх$/$сверху.}\end{itemize}
}
\item{
public static final Direction DOWN\begin{itemize}\item{\vskip -.9ex Вниз$/$снизу.}\end{itemize}
}
\end{itemize}
}
\startsubsubsection{Methods}{
\vskip -2em
\begin{itemize}
\item{\vskip -1.9ex 
\membername{valueOf}
{\tt public static Direction {\bf valueOf}( {\tt String } {\bf name} )
\label{l73}\label{l74}}%end signature
}%end item
\divideents{values}
\item{\vskip -1.9ex 
\membername{values}
{\tt public static Direction[] {\bf values}(  )
\label{l75}\label{l76}}%end signature
}%end item
\end{itemize}
}
\hide{inherited}{
\startsubsubsection{Methods inherited from class {\tt Enum}}{
\par{\small 
\refdefined{l24}\vskip -2em
\begin{itemize}
\item{\vskip -1.9ex 
\membername{clone}
{\tt protected final Object {\bf clone}(  )
}%end signature
}%end item
\divideents{compareTo}
\item{\vskip -1.9ex 
\membername{compareTo}
{\tt public final int {\bf compareTo}( {\tt Enum } {\bf arg0} )
}%end signature
}%end item
\divideents{equals}
\item{\vskip -1.9ex 
\membername{equals}
{\tt public final boolean {\bf equals}( {\tt Object } {\bf arg0} )
}%end signature
}%end item
\divideents{finalize}
\item{\vskip -1.9ex 
\membername{finalize}
{\tt protected final void {\bf finalize}(  )
}%end signature
}%end item
\divideents{getDeclaringClass}
\item{\vskip -1.9ex 
\membername{getDeclaringClass}
{\tt public final Class {\bf getDeclaringClass}(  )
}%end signature
}%end item
\divideents{hashCode}
\item{\vskip -1.9ex 
\membername{hashCode}
{\tt public final int {\bf hashCode}(  )
}%end signature
}%end item
\divideents{name}
\item{\vskip -1.9ex 
\membername{name}
{\tt public final String {\bf name}(  )
}%end signature
}%end item
\divideents{ordinal}
\item{\vskip -1.9ex 
\membername{ordinal}
{\tt public final int {\bf ordinal}(  )
}%end signature
}%end item
\divideents{toString}
\item{\vskip -1.9ex 
\membername{toString}
{\tt public String {\bf toString}(  )
}%end signature
}%end item
\divideents{valueOf}
\item{\vskip -1.9ex 
\membername{valueOf}
{\tt public static Enum {\bf valueOf}( {\tt Class } {\bf arg0},
{\tt String } {\bf arg1} )
}%end signature
}%end item
\end{itemize}
}}
}
}
\startsection{Class}{Game}{l6}{%
{\small Предоставляет доступ к различным игровым константам.}
\vskip .1in 
\startsubsubsection{Declaration}{
\fbox{\vbox{
\hbox{\vbox{\small public 
class 
Game}}
\noindent\hbox{\vbox{{\bf extends} Object}}
}}}

% Removed by DocsPostProcessor:
% \startsubsubsection{Constructors}{
% \vskip -2em
% \begin{itemize}
% \item{\vskip -1.9ex 
% \membername{Game}
% {\tt public {\bf Game}( {\tt long } {\bf randomSeed},
% {\tt int } {\bf tickCount},
% {\tt int } {\bf worldWidth},
% {\tt int } {\bf worldHeight},
% {\tt double } {\bf trackTileSize},
% {\tt double } {\bf trackTileMargin},
% {\tt int } {\bf lapCount},
% {\tt int } {\bf lapTickCount},
% {\tt int } {\bf initialFreezeDurationTicks},
% {\tt double } {\bf burningTimeDurationFactor},
% {\tt int[]} {\bf finishTrackScores},
% {\tt int } {\bf finishLapScore},
% {\tt double } {\bf lapWaypointsSummaryScoreFactor},
% {\tt double } {\bf carDamageScoreFactor},
% {\tt int } {\bf carEliminationScore},
% {\tt double } {\bf carWidth},
% {\tt double } {\bf carHeight},
% {\tt double } {\bf carEnginePowerChangePerTick},
% {\tt double } {\bf carWheelTurnChangePerTick},
% {\tt double } {\bf carAngularSpeedFactor},
% {\tt double } {\bf carMovementAirFrictionFactor},
% {\tt double } {\bf carRotationAirFrictionFactor},
% {\tt double } {\bf carLengthwiseMovementFrictionFactor},
% {\tt double } {\bf carCrosswiseMovementFrictionFactor},
% {\tt double } {\bf carRotationFrictionFactor},
% {\tt int } {\bf throwProjectileCooldownTicks},
% {\tt int } {\bf useNitroCooldownTicks},
% {\tt int } {\bf spillOilCooldownTicks},
% {\tt double } {\bf nitroEnginePowerFactor},
% {\tt int } {\bf nitroDurationTicks},
% {\tt int } {\bf carReactivationTimeTicks},
% {\tt double } {\bf buggyMass},
% {\tt double } {\bf buggyEngineForwardPower},
% {\tt double } {\bf buggyEngineRearPower},
% {\tt double } {\bf jeepMass},
% {\tt double } {\bf jeepEngineForwardPower},
% {\tt double } {\bf jeepEngineRearPower},
% {\tt double } {\bf bonusSize},
% {\tt double } {\bf bonusMass},
% {\tt int } {\bf pureScoreAmount},
% {\tt double } {\bf washerRadius},
% {\tt double } {\bf washerMass},
% {\tt double } {\bf washerInitialSpeed},
% {\tt double } {\bf washerDamage},
% {\tt double } {\bf sideWasherAngle},
% {\tt double } {\bf tireRadius},
% {\tt double } {\bf tireMass},
% {\tt double } {\bf tireInitialSpeed},
% {\tt double } {\bf tireDamageFactor},
% {\tt double } {\bf tireDisappearSpeedFactor},
% {\tt double } {\bf oilSlickInitialRange},
% {\tt double } {\bf oilSlickRadius},
% {\tt int } {\bf oilSlickLifetime},
% {\tt int } {\bf maxOiledStateDurationTicks} )
% \label{l77}\label{l78}}%end signature
% }%end item
% \end{itemize}
% }
% \\ Removed by DocsPostProcessor.

\startsubsubsection{Methods}{
\vskip -2em
\begin{itemize}
\item{\vskip -1.9ex 
\membername{getBonusMass}
{\tt public double {\bf getBonusMass}(  )
\label{l79}\label{l80}}%end signature
\begin{itemize}
\sld
\item{{\bf Returns} - 
Возвращает массу бонуса. 
}%end item
\end{itemize}
}%end item
\divideents{getBonusSize}
\item{\vskip -1.9ex 
\membername{getBonusSize}
{\tt public double {\bf getBonusSize}(  )
\label{l81}\label{l82}}%end signature
\begin{itemize}
\sld
\item{{\bf Returns} - 
Возвращает размер (ширину и высоту) бонуса. 
}%end item
\end{itemize}
}%end item
\divideents{getBuggyEngineForwardPower}
\item{\vskip -1.9ex 
\membername{getBuggyEngineForwardPower}
{\tt public double {\bf getBuggyEngineForwardPower}(  )
\label{l83}\label{l84}}%end signature
\begin{itemize}
\sld
\item{{\bf Returns} - 
Возвращает максимальную мощность двигателя кодемобиля типа багги ({\tt CarType.BUGGY}) в направлении,
 совпадающем с направлением кодемобиля. 
}%end item
\end{itemize}
}%end item
\divideents{getBuggyEngineRearPower}
\item{\vskip -1.9ex 
\membername{getBuggyEngineRearPower}
{\tt public double {\bf getBuggyEngineRearPower}(  )
\label{l85}\label{l86}}%end signature
\begin{itemize}
\sld
\item{{\bf Returns} - 
Возвращает максимальную мощность двигателя кодемобиля типа багги ({\tt CarType.BUGGY}) в направлении,
 противоположном направлению кодемобиля. 
}%end item
\end{itemize}
}%end item
\divideents{getBuggyMass}
\item{\vskip -1.9ex 
\membername{getBuggyMass}
{\tt public double {\bf getBuggyMass}(  )
\label{l87}\label{l88}}%end signature
\begin{itemize}
\sld
\item{{\bf Returns} - 
Возвращает массу кодемобиля типа багги ({\tt CarType.BUGGY}). 
}%end item
\end{itemize}
}%end item
\divideents{getBurningTimeDurationFactor}
\item{\vskip -1.9ex 
\membername{getBurningTimeDurationFactor}
{\tt public double {\bf getBurningTimeDurationFactor}(  )
\label{l89}\label{l90}}%end signature
\begin{itemize}
\sld
\item{{\bf Returns} - 
Возвращает коэффициент, определяющий количество тиков до завершения игры после финиширования трассы
 очередным кодемобилем. Для получения более подробной информации смотрите документацию к
 {\tt world.lastTickIndex}. 
}%end item
\end{itemize}
}%end item
\divideents{getCarAngularSpeedFactor}
\item{\vskip -1.9ex 
\membername{getCarAngularSpeedFactor}
{\tt public double {\bf getCarAngularSpeedFactor}(  )
\label{l91}\label{l92}}%end signature
\begin{itemize}
\sld
\item{{\bf Returns} - 
Возвращает коэффициент, используемый для вычисления составляющей угловой скорости кодемобиля, порождаемой
 движением кодемобиля при ненулевом относительном угле поворота колёс. Для получения более подробной информации
 смотрите документацию к {\tt move.wheelTurn}. 
}%end item
\end{itemize}
}%end item
\divideents{getCarCrosswiseMovementFrictionFactor}
\item{\vskip -1.9ex 
\membername{getCarCrosswiseMovementFrictionFactor}
{\tt public double {\bf getCarCrosswiseMovementFrictionFactor}(  )
\label{l93}\label{l94}}%end signature
\begin{itemize}
\sld
\item{{\bf Returns} - 
Возвращает абсолютную потерю составляющей скорости кодемобиля, направленной вдоль поперечной оси
 кодемобиля, за один тик. 
}%end item
\end{itemize}
}%end item
\divideents{getCarDamageScoreFactor}
\item{\vskip -1.9ex 
\membername{getCarDamageScoreFactor}
{\tt public double {\bf getCarDamageScoreFactor}(  )
\label{l95}\label{l96}}%end signature
\begin{itemize}
\sld
\item{{\bf Returns} - 
Возвращает количество баллов, зарабатываемых кодемобилем при нанесении {\tt 1.0} урона кодемобилю
 другого игрока. При нанесении меньшего урона количество баллов пропорционально падает. Результат всегда
 округляется в меньшую сторону. 
}%end item
\end{itemize}
}%end item
\divideents{getCarEliminationScore}
\item{\vskip -1.9ex 
\membername{getCarEliminationScore}
{\tt public int {\bf getCarEliminationScore}(  )
\label{l97}\label{l98}}%end signature
\begin{itemize}
\sld
\item{{\bf Returns} - 
Возвращает количество баллов, зарабатываемых кодемобилем при уничтожении кодемобиля другого игрока. 
}%end item
\end{itemize}
}%end item
\divideents{getCarEnginePowerChangePerTick}
\item{\vskip -1.9ex 
\membername{getCarEnginePowerChangePerTick}
{\tt public double {\bf getCarEnginePowerChangePerTick}(  )
\label{l99}\label{l100}}%end signature
\begin{itemize}
\sld
\item{{\bf Returns} - 
Возвращает максимальное значение, на которое может измениться относительная мощность двигателя кодемобиля
 ({\tt car.enginePower}) за один тик. 
}%end item
\end{itemize}
}%end item
\divideents{getCarHeight}
\item{\vskip -1.9ex 
\membername{getCarHeight}
{\tt public double {\bf getCarHeight}(  )
\label{l101}\label{l102}}%end signature
\begin{itemize}
\sld
\item{{\bf Returns} - 
Возвращает высоту кодемобиля. 
}%end item
\end{itemize}
}%end item
\divideents{getCarLengthwiseMovementFrictionFactor}
\item{\vskip -1.9ex 
\membername{getCarLengthwiseMovementFrictionFactor}
{\tt public double {\bf getCarLengthwiseMovementFrictionFactor}(  )
\label{l103}\label{l104}}%end signature
\begin{itemize}
\sld
\item{{\bf Returns} - 
Возвращает абсолютную потерю составляющей скорости кодемобиля, направленной вдоль продольной оси
 кодемобиля, за один тик. 
}%end item
\end{itemize}
}%end item
\divideents{getCarMovementAirFrictionFactor}
\item{\vskip -1.9ex 
\membername{getCarMovementAirFrictionFactor}
{\tt public double {\bf getCarMovementAirFrictionFactor}(  )
\label{l105}\label{l106}}%end signature
\begin{itemize}
\sld
\item{{\bf Returns} - 
Возвращает относительную потерю модуля скорости кодемобиля за один тик. 
}%end item
\end{itemize}
}%end item
\divideents{getCarReactivationTimeTicks}
\item{\vskip -1.9ex 
\membername{getCarReactivationTimeTicks}
{\tt public int {\bf getCarReactivationTimeTicks}(  )
\label{l107}\label{l108}}%end signature
\begin{itemize}
\sld
\item{{\bf Returns} - 
Возвращает длительность интервала в тиках, по прошествии которого сильно повреждённый кодемобиль
 (значение {\tt car.durability} равно нулю) будет восстановлен. 
}%end item
\end{itemize}
}%end item
\divideents{getCarRotationAirFrictionFactor}
\item{\vskip -1.9ex 
\membername{getCarRotationAirFrictionFactor}
{\tt public double {\bf getCarRotationAirFrictionFactor}(  )
\label{l109}\label{l110}}%end signature
\begin{itemize}
\sld
\item{{\bf Returns} - 
Возвращает относительную потерю модуля угловой скорости кодемобиля за один тик. Относительная потеря
 применяется только к составляющей угловой скорости кодемобиля, не порождаемой движением кодемобиля при ненулевом
 относительном угле поворота колёс. Для получения более подробной информации смотрите документацию к
 {\tt move.wheelTurn}. 
}%end item
\end{itemize}
}%end item
\divideents{getCarRotationFrictionFactor}
\item{\vskip -1.9ex 
\membername{getCarRotationFrictionFactor}
{\tt public double {\bf getCarRotationFrictionFactor}(  )
\label{l111}\label{l112}}%end signature
\begin{itemize}
\sld
\item{{\bf Returns} - 
Возвращает абсолютную потерю модуля угловой скорости кодемобиля за один тик. Абсолютная потеря
 применяется только к составляющей угловой скорости кодемобиля, не порождаемой движением кодемобиля при ненулевом
 относительном угле поворота колёс. Для получения более подробной информации смотрите документацию к
 {\tt move.wheelTurn}. 
}%end item
\end{itemize}
}%end item
\divideents{getCarWheelTurnChangePerTick}
\item{\vskip -1.9ex 
\membername{getCarWheelTurnChangePerTick}
{\tt public double {\bf getCarWheelTurnChangePerTick}(  )
\label{l113}\label{l114}}%end signature
\begin{itemize}
\sld
\item{{\bf Returns} - 
Возвращает максимальное значение, на которое может измениться относительный угол поворота колёс
 кодемобиля ({\tt car.wheelTurn}) за один тик. 
}%end item
\end{itemize}
}%end item
\divideents{getCarWidth}
\item{\vskip -1.9ex 
\membername{getCarWidth}
{\tt public double {\bf getCarWidth}(  )
\label{l115}\label{l116}}%end signature
\begin{itemize}
\sld
\item{{\bf Returns} - 
Возвращает ширину кодемобиля. 
}%end item
\end{itemize}
}%end item
\divideents{getFinishLapScore}
\item{\vskip -1.9ex 
\membername{getFinishLapScore}
{\tt public int {\bf getFinishLapScore}(  )
\label{l117}\label{l118}}%end signature
\begin{itemize}
\sld
\item{{\bf Returns} - 
Возвращает количество баллов, зарабатываемых кодемобилем при прохождении одного круга.
 Баллы начисляются не единовременно, а постепенно, по мере прохождения ключевых точек. 
}%end item
\end{itemize}
}%end item
\divideents{getFinishTrackScores}
\item{\vskip -1.9ex 
\membername{getFinishTrackScores}
{\tt public int[] {\bf getFinishTrackScores}(  )
\label{l119}\label{l120}}%end signature
\begin{itemize}
\sld
\item{{\bf Returns} - 
Возвращает 0-индексированный массив, содержащий количество баллов, зарабатываемых кодемобилями при
 завершении трассы. Кодемобиль, финишировавший первым, приносит владельцу {\tt finishTrackScores[0]} баллов,
 вторым --- {\tt finishTrackScores[1]} и так далее. 
}%end item
\end{itemize}
}%end item
\divideents{getInitialFreezeDurationTicks}
\item{\vskip -1.9ex 
\membername{getInitialFreezeDurationTicks}
{\tt public int {\bf getInitialFreezeDurationTicks}(  )
\label{l121}\label{l122}}%end signature
\begin{itemize}
\sld
\item{{\bf Returns} - 
Возвращает количество тиков в начале игры, в течение которых кодемобиль не может изменять своё положение.
 Значение является составной частью выражения для нахождения базовой длительности игры ({\tt game.tickCount}). 
}%end item
\end{itemize}
}%end item
\divideents{getJeepEngineForwardPower}
\item{\vskip -1.9ex 
\membername{getJeepEngineForwardPower}
{\tt public double {\bf getJeepEngineForwardPower}(  )
\label{l123}\label{l124}}%end signature
\begin{itemize}
\sld
\item{{\bf Returns} - 
Возвращает максимальную мощность двигателя кодемобиля типа джип ({\tt CarType.JEEP}) в направлении,
 совпадающем с направлением кодемобиля. 
}%end item
\end{itemize}
}%end item
\divideents{getJeepEngineRearPower}
\item{\vskip -1.9ex 
\membername{getJeepEngineRearPower}
{\tt public double {\bf getJeepEngineRearPower}(  )
\label{l125}\label{l126}}%end signature
\begin{itemize}
\sld
\item{{\bf Returns} - 
Возвращает максимальную мощность двигателя кодемобиля типа джип ({\tt CarType.JEEP}) в направлении,
 противоположном направлению кодемобиля. 
}%end item
\end{itemize}
}%end item
\divideents{getJeepMass}
\item{\vskip -1.9ex 
\membername{getJeepMass}
{\tt public double {\bf getJeepMass}(  )
\label{l127}\label{l128}}%end signature
\begin{itemize}
\sld
\item{{\bf Returns} - 
Возвращает массу кодемобиля типа джип ({\tt CarType.JEEP}). 
}%end item
\end{itemize}
}%end item
\divideents{getLapCount}
\item{\vskip -1.9ex 
\membername{getLapCount}
{\tt public int {\bf getLapCount}(  )
\label{l129}\label{l130}}%end signature
\begin{itemize}
\sld
\item{{\bf Returns} - 
Возвращает количество кругов (циклов прохождения списка ключевых точек {\tt world.waypoints}),
 которое необходимо пройти для завершения трассы. 
}%end item
\end{itemize}
}%end item
\divideents{getLapTickCount}
\item{\vskip -1.9ex 
\membername{getLapTickCount}
{\tt public int {\bf getLapTickCount}(  )
\label{l131}\label{l132}}%end signature
\begin{itemize}
\sld
\item{{\bf Returns} - 
Возвращает количество тиков, которое выделяется кодемобилям на прохождение одного круга.
 Значение является составной частью выражения для нахождения базовой длительности игры ({\tt game.tickCount}) и
 не используется в целях ограничения на прохождение одного отдельного круга. 
}%end item
\end{itemize}
}%end item
\divideents{getLapWaypointsSummaryScoreFactor}
\item{\vskip -1.9ex 
\membername{getLapWaypointsSummaryScoreFactor}
{\tt public double {\bf getLapWaypointsSummaryScoreFactor}(  )
\label{l133}\label{l134}}%end signature
\begin{itemize}
\sld
\item{{\bf Returns} - 
Возвращает долю от баллов за круг ({\tt game.finishLapScore}), которую кодемобиль заработает при
 прохождении всех ключевых точек круга, кроме последней. Баллы равномерно распределены по ключевым точкам. 
}%end item
\end{itemize}
}%end item
\divideents{getMaxOiledStateDurationTicks}
\item{\vskip -1.9ex 
\membername{getMaxOiledStateDurationTicks}
{\tt public int {\bf getMaxOiledStateDurationTicks}(  )
\label{l135}\label{l136}}%end signature
\begin{itemize}
\sld
\item{{\bf Returns} - 
Возвращает максимально возможную длительность высыхания кодемобиля, центр которого попал в лужу мазута.
 При этом, длительность высыхания лужа мазута сокращается на то же количество тиков. Таким образом, реальная
 длительность высыхания кодемобиля не может превышать оставшуюся длительность высыхания лужи. 
}%end item
\end{itemize}
}%end item
\divideents{getNitroDurationTicks}
\item{\vskip -1.9ex 
\membername{getNitroDurationTicks}
{\tt public int {\bf getNitroDurationTicks}(  )
\label{l137}\label{l138}}%end signature
\begin{itemize}
\sld
\item{{\bf Returns} - 
Возвращает длительность ускорения \textless \textless нитро\textgreater \textgreater  в тиках. 
}%end item
\end{itemize}
}%end item
\divideents{getNitroEnginePowerFactor}
\item{\vskip -1.9ex 
\membername{getNitroEnginePowerFactor}
{\tt public double {\bf getNitroEnginePowerFactor}(  )
\label{l139}\label{l140}}%end signature
\begin{itemize}
\sld
\item{{\bf Returns} - 
Возвращает относительную мощность двигателя кодемобиля, мгновенно устанавливаемую при использовании
 системы закиси азота для ускорения кодемобиля. 
}%end item
\end{itemize}
}%end item
\divideents{getOilSlickInitialRange}
\item{\vskip -1.9ex 
\membername{getOilSlickInitialRange}
{\tt public double {\bf getOilSlickInitialRange}(  )
\label{l141}\label{l142}}%end signature
\begin{itemize}
\sld
\item{{\bf Returns} - 
Возвращает расстояние между ближайшими точками лужи мазута и кодемобиля при использовании канистры с
 мазутом. 
}%end item
\end{itemize}
}%end item
\divideents{getOilSlickLifetime}
\item{\vskip -1.9ex 
\membername{getOilSlickLifetime}
{\tt public int {\bf getOilSlickLifetime}(  )
\label{l143}\label{l144}}%end signature
\begin{itemize}
\sld
\item{{\bf Returns} - 
Возвращает длительность высыхания лужи мазута в тиках. 
}%end item
\end{itemize}
}%end item
\divideents{getOilSlickRadius}
\item{\vskip -1.9ex 
\membername{getOilSlickRadius}
{\tt public double {\bf getOilSlickRadius}(  )
\label{l145}\label{l146}}%end signature
\begin{itemize}
\sld
\item{{\bf Returns} - 
Возвращает радиус лужи мазута. 
}%end item
\end{itemize}
}%end item
\divideents{getPureScoreAmount}
\item{\vskip -1.9ex 
\membername{getPureScoreAmount}
{\tt public int {\bf getPureScoreAmount}(  )
\label{l147}\label{l148}}%end signature
\begin{itemize}
\sld
\item{{\bf Returns} - 
Возвращает количество баллов, мгновенно получаемых игроком, кодемобиль которого подобрал бонусные баллы
 ({\tt BonusType.PURE\_SCORE}). 
}%end item
\end{itemize}
}%end item
\divideents{getRandomSeed}
\item{\vskip -1.9ex 
\membername{getRandomSeed}
{\tt public long {\bf getRandomSeed}(  )
\label{l149}\label{l150}}%end signature
\begin{itemize}
\sld
\item{{\bf Returns} - 
Возвращает некоторое число, которое ваша стратегия может использовать для инициализации генератора
 случайных чисел. Данное значение имеет рекомендательный характер, однако позволит более точно
 воспроизводить прошедшие игры. 
}%end item
\end{itemize}
}%end item
\divideents{getSideWasherAngle}
\item{\vskip -1.9ex 
\membername{getSideWasherAngle}
{\tt public double {\bf getSideWasherAngle}(  )
\label{l151}\label{l152}}%end signature
\begin{itemize}
\sld
\item{{\bf Returns} - 
Возвращает модуль отклонения направления полёта двух шайб от направления кодемобиля.
 Направление третьей шайбы совпадает с направлением кодемобиля. 
}%end item
\end{itemize}
}%end item
\divideents{getSpillOilCooldownTicks}
\item{\vskip -1.9ex 
\membername{getSpillOilCooldownTicks}
{\tt public int {\bf getSpillOilCooldownTicks}(  )
\label{l153}\label{l154}}%end signature
\begin{itemize}
\sld
\item{{\bf Returns} - 
Возвращает длительность задержки в тиках, применяемой к кодемобилю после использования им канистры с
 мазутом. В течение этого времени кодемобиль не может разлить ещё одну канистру. 
}%end item
\end{itemize}
}%end item
\divideents{getThrowProjectileCooldownTicks}
\item{\vskip -1.9ex 
\membername{getThrowProjectileCooldownTicks}
{\tt public int {\bf getThrowProjectileCooldownTicks}(  )
\label{l155}\label{l156}}%end signature
\begin{itemize}
\sld
\item{{\bf Returns} - 
Возвращает длительность задержки в тиках, применяемой к кодемобилю после метания им снаряда.
 В течение этого времени кодемобиль не может метать новые снаряды. 
}%end item
\end{itemize}
}%end item
\divideents{getTickCount}
\item{\vskip -1.9ex 
\membername{getTickCount}
{\tt public int {\bf getTickCount}(  )
\label{l157}\label{l158}}%end signature
\begin{itemize}
\sld
\item{{\bf Returns} - 
Возвращает базовую длительность игры в тиках.
 Реальная длительность может отличаться от этого значения в меньшую сторону.
 Поле может быть определено как {\tt game.initialFreezeDurationTicks + game.lapCount * game.lapTickCount}.
 Значение поля не меняется в процессе игры. Эквивалентно {\tt world.tickCount}. 
}%end item
\end{itemize}
}%end item
\divideents{getTireDamageFactor}
\item{\vskip -1.9ex 
\membername{getTireDamageFactor}
{\tt public double {\bf getTireDamageFactor}(  )
\label{l159}\label{l160}}%end signature
\begin{itemize}
\sld
\item{{\bf Returns} - 
Возвращает количество урона, которое шина нанесёт неподвижно стоящему кодемобилю при попадании в него с
 начальной скоростью ({\tt game.tireInitialSpeed}) и под прямым углом к поверхности кодемобиля. Движение
 кодемобиля в направлении, совпадающем с направлением движения шины, уменьшает урон, движение в противоположном
 направлении --- увеличивает. 
}%end item
\end{itemize}
}%end item
\divideents{getTireDisappearSpeedFactor}
\item{\vskip -1.9ex 
\membername{getTireDisappearSpeedFactor}
{\tt public double {\bf getTireDisappearSpeedFactor}(  )
\label{l161}\label{l162}}%end signature
\begin{itemize}
\sld
\item{{\bf Returns} - 
Возвращает отношение текущей скорости шины к начальной ({\tt game.tireInitialSpeed}), при превышении
 которого в момент столкновения с другим объектом шина отскакивает и продолжает свой полёт. В противном случае
 шина убирается из игрового мира. 
}%end item
\end{itemize}
}%end item
\divideents{getTireInitialSpeed}
\item{\vskip -1.9ex 
\membername{getTireInitialSpeed}
{\tt public double {\bf getTireInitialSpeed}(  )
\label{l163}\label{l164}}%end signature
\begin{itemize}
\sld
\item{{\bf Returns} - 
Возвращает начальную скорость шины ({\tt ProjectileType.TIRE}). 
}%end item
\end{itemize}
}%end item
\divideents{getTireMass}
\item{\vskip -1.9ex 
\membername{getTireMass}
{\tt public double {\bf getTireMass}(  )
\label{l165}\label{l166}}%end signature
\begin{itemize}
\sld
\item{{\bf Returns} - 
Возвращает массу шины ({\tt ProjectileType.TIRE}). 
}%end item
\end{itemize}
}%end item
\divideents{getTireRadius}
\item{\vskip -1.9ex 
\membername{getTireRadius}
{\tt public double {\bf getTireRadius}(  )
\label{l167}\label{l168}}%end signature
\begin{itemize}
\sld
\item{{\bf Returns} - 
Возвращает радиус шины ({\tt ProjectileType.TIRE}). 
}%end item
\end{itemize}
}%end item
\divideents{getTrackTileMargin}
\item{\vskip -1.9ex 
\membername{getTrackTileMargin}
{\tt public double {\bf getTrackTileMargin}(  )
\label{l169}\label{l170}}%end signature
\begin{itemize}
\sld
\item{{\bf Returns} - 
Возвращает отступ от границы тайла до границы прямого участка трассы, проходящего через этот тайл.
 Радиусы всех закруглённых сочленений участков трассы также равны этому значению. 
}%end item
\end{itemize}
}%end item
\divideents{getTrackTileSize}
\item{\vskip -1.9ex 
\membername{getTrackTileSize}
{\tt public double {\bf getTrackTileSize}(  )
\label{l171}\label{l172}}%end signature
\begin{itemize}
\sld
\item{{\bf Returns} - 
Возвращает размер (ширину и высоту) одного тайла. 
}%end item
\end{itemize}
}%end item
\divideents{getUseNitroCooldownTicks}
\item{\vskip -1.9ex 
\membername{getUseNitroCooldownTicks}
{\tt public int {\bf getUseNitroCooldownTicks}(  )
\label{l173}\label{l174}}%end signature
\begin{itemize}
\sld
\item{{\bf Returns} - 
Возвращает длительность задержки в тиках, применяемой к кодемобилю после использования им ускорения
 \textless \textless нитро\textgreater \textgreater . В течение этого времени кодемобиль не может повторно использовать систему закиси азота. 
}%end item
\end{itemize}
}%end item
\divideents{getWasherDamage}
\item{\vskip -1.9ex 
\membername{getWasherDamage}
{\tt public double {\bf getWasherDamage}(  )
\label{l175}\label{l176}}%end signature
\begin{itemize}
\sld
\item{{\bf Returns} - 
Возвращает урон шайбы ({\tt ProjectileType.WASHER}). 
}%end item
\end{itemize}
}%end item
\divideents{getWasherInitialSpeed}
\item{\vskip -1.9ex 
\membername{getWasherInitialSpeed}
{\tt public double {\bf getWasherInitialSpeed}(  )
\label{l177}\label{l178}}%end signature
\begin{itemize}
\sld
\item{{\bf Returns} - 
Возвращает начальную скорость шайбы ({\tt ProjectileType.WASHER}). 
}%end item
\end{itemize}
}%end item
\divideents{getWasherMass}
\item{\vskip -1.9ex 
\membername{getWasherMass}
{\tt public double {\bf getWasherMass}(  )
\label{l179}\label{l180}}%end signature
\begin{itemize}
\sld
\item{{\bf Returns} - 
Возвращает массу шайбы ({\tt ProjectileType.WASHER}). 
}%end item
\end{itemize}
}%end item
\divideents{getWasherRadius}
\item{\vskip -1.9ex 
\membername{getWasherRadius}
{\tt public double {\bf getWasherRadius}(  )
\label{l181}\label{l182}}%end signature
\begin{itemize}
\sld
\item{{\bf Returns} - 
Возвращает радиус шайбы ({\tt ProjectileType.WASHER}). 
}%end item
\end{itemize}
}%end item
\divideents{getWorldHeight}
\item{\vskip -1.9ex 
\membername{getWorldHeight}
{\tt public int {\bf getWorldHeight}(  )
\label{l183}\label{l184}}%end signature
\begin{itemize}
\sld
\item{{\bf Returns} - 
Возвращает высоту игрового мира в тайлах. 
}%end item
\end{itemize}
}%end item
\divideents{getWorldWidth}
\item{\vskip -1.9ex 
\membername{getWorldWidth}
{\tt public int {\bf getWorldWidth}(  )
\label{l185}\label{l186}}%end signature
\begin{itemize}
\sld
\item{{\bf Returns} - 
Возвращает ширину игрового мира в тайлах. 
}%end item
\end{itemize}
}%end item
\end{itemize}
}
\hide{inherited}{
}
}
\startsection{Class}{Move}{l7}{%
{\small Стратегия игрока может управлять кодемобилем посредством установки свойств объекта данного класса.}
\vskip .1in 
\startsubsubsection{Declaration}{
\fbox{\vbox{
\hbox{\vbox{\small public 
class 
Move}}
\noindent\hbox{\vbox{{\bf extends} Object}}
}}}

% Removed by DocsPostProcessor:
% \startsubsubsection{Constructors}{
% \vskip -2em
% \begin{itemize}
% \item{\vskip -1.9ex 
% \membername{Move}
% {\tt public {\bf Move}(  )
% \label{l187}\label{l188}}%end signature
% }%end item
% \end{itemize}
% }
% \\ Removed by DocsPostProcessor.

\startsubsubsection{Methods}{
\vskip -2em
\begin{itemize}
\item{\vskip -1.9ex 
\membername{getEnginePower}
{\tt public double {\bf getEnginePower}(  )
\label{l189}\label{l190}}%end signature
\begin{itemize}
\sld
\item{{\bf Returns} - 
Возвращает текущую установку режима работы двигателя кодемобиля. 
}%end item
\end{itemize}
}%end item
\divideents{getWheelTurn}
\item{\vskip -1.9ex 
\membername{getWheelTurn}
{\tt public double {\bf getWheelTurn}(  )
\label{l191}\label{l192}}%end signature
\begin{itemize}
\sld
\item{{\bf Returns} - 
Возвращает текущий относительный угол поворота колёс кодемобиля. 
}%end item
\end{itemize}
}%end item
\divideents{isBrake}
\item{\vskip -1.9ex 
\membername{isBrake}
{\tt public boolean {\bf isBrake}(  )
\label{l193}\label{l194}}%end signature
\begin{itemize}
\sld
\item{{\bf Returns} - 
Возвращает текущее положение педали тормоза. 
}%end item
\end{itemize}
}%end item
\divideents{isSpillOil}
\item{\vskip -1.9ex 
\membername{isSpillOil}
{\tt public boolean {\bf isSpillOil}(  )
\label{l195}\label{l196}}%end signature
\begin{itemize}
\sld
\item{{\bf Returns} - 
Возвращает текущее значение указания разлить канистру с мазутом. 
}%end item
\end{itemize}
}%end item
\divideents{isThrowProjectile}
\item{\vskip -1.9ex 
\membername{isThrowProjectile}
{\tt public boolean {\bf isThrowProjectile}(  )
\label{l197}\label{l198}}%end signature
\begin{itemize}
\sld
\item{{\bf Returns} - 
Возвращает текущее значение указания метнуть снаряд. 
}%end item
\end{itemize}
}%end item
\divideents{isUseNitro}
\item{\vskip -1.9ex 
\membername{isUseNitro}
{\tt public boolean {\bf isUseNitro}(  )
\label{l199}\label{l200}}%end signature
\begin{itemize}
\sld
\item{{\bf Returns} - 
Возвращает текущее значение указания использовать \textless \textless нитро\textgreater \textgreater . 
}%end item
\end{itemize}
}%end item
\divideents{setBrake}
\item{\vskip -1.9ex 
\membername{setBrake}
{\tt public void {\bf setBrake}( {\tt boolean } {\bf brake} )
\label{l201}\label{l202}}%end signature
\begin{itemize}
\sld
\item{
\sld
{\bf Usage}
  \begin{itemize}\isep
   \item{
Задаёт текущее положение педали тормоза.
 
% Removed by DocsPostProcessor:
% \textless p$/$\textgreater
% \\ Removed by DocsPostProcessor.
 
 При утопленной педали тормоза значение силы трения вдоль направления, совпадающего с углом поворота кодемобиля,
 возрастает с {\tt game.carLengthwiseMovementFrictionFactor} до {\tt game.carCrosswiseMovementFrictionFactor}.
}%end item
  \end{itemize}
}
\end{itemize}
}%end item
\divideents{setEnginePower}
\item{\vskip -1.9ex 
\membername{setEnginePower}
{\tt public void {\bf setEnginePower}( {\tt double } {\bf enginePower} )
\label{l203}\label{l204}}%end signature
\begin{itemize}
\sld
\item{
\sld
{\bf Usage}
  \begin{itemize}\isep
   \item{
Задаёт установку режима работы двигателя кодемобиля.
 
% Removed by DocsPostProcessor:
% \textless p$/$\textgreater
% \\ Removed by DocsPostProcessor.
 
 Установка режима работы является относительной и должна лежать в интервале от {\tt -1.0} до {\tt 1.0}.
 Значения, выходящие за указанный интервал, будут приведены к ближайшей его границе.
 
% Removed by DocsPostProcessor:
% \textless p$/$\textgreater
% \\ Removed by DocsPostProcessor.
 
 Реальный режим работы двигателя может отличаться от установки, так как его изменение происходит не мгновенно, а
 со скоростью не более {\tt game.carEnginePowerChangePerTick} за тик. Режим работы двигателя фактически
 определяет ускорение в направлении, совпадающем с углом поворота кодемобиля. Абсолютное значение ускорения
 равномерно изменяется на интервале от {\tt -1.0} до {\tt 0.0} и на интервале от {\tt 0.0} до {\tt 1.0}.
}%end item
  \end{itemize}
}
\end{itemize}
}%end item
\divideents{setSpillOil}
\item{\vskip -1.9ex 
\membername{setSpillOil}
{\tt public void {\bf setSpillOil}( {\tt boolean } {\bf spillOil} )
\label{l205}\label{l206}}%end signature
\begin{itemize}
\sld
\item{
\sld
{\bf Usage}
  \begin{itemize}\isep
   \item{
Устанавливает значение указания разлить канистру с мазутом.
 
% Removed by DocsPostProcessor:
% \textless p$/$\textgreater
% \\ Removed by DocsPostProcessor.
 
 Указание может быть проигнорировано, если у кодемобиля не осталось канистр с мазутом
 либо прошло менее {\tt game.spillOilCooldownTicks} тиков с момента предыдущего использования данного действия.
}%end item
  \end{itemize}
}
\end{itemize}
}%end item
\divideents{setThrowProjectile}
\item{\vskip -1.9ex 
\membername{setThrowProjectile}
{\tt public void {\bf setThrowProjectile}( {\tt boolean } {\bf throwProjectile} )
\label{l207}\label{l208}}%end signature
\begin{itemize}
\sld
\item{
\sld
{\bf Usage}
  \begin{itemize}\isep
   \item{
Устанавливает значение указания метнуть снаряд.
 
% Removed by DocsPostProcessor:
% \textless p$/$\textgreater
% \\ Removed by DocsPostProcessor.
 
 Указание может быть проигнорировано, если у кодемобиля не осталось снарядов
 либо прошло менее {\tt game.throwProjectileCooldownTicks} тиков с момента запуска предыдущего снаряда.
}%end item
  \end{itemize}
}
\end{itemize}
}%end item
\divideents{setUseNitro}
\item{\vskip -1.9ex 
\membername{setUseNitro}
{\tt public void {\bf setUseNitro}( {\tt boolean } {\bf useNitro} )
\label{l209}\label{l210}}%end signature
\begin{itemize}
\sld
\item{
\sld
{\bf Usage}
  \begin{itemize}\isep
   \item{
Устанавливает значение указания использовать \textless \textless нитро\textgreater \textgreater .
 
% Removed by DocsPostProcessor:
% \textless p$/$\textgreater
% \\ Removed by DocsPostProcessor.
 
 Указание может быть проигнорировано, если у кодемобиля не осталось зарядов для системы закиси азота
 либо прошло менее {\tt game.useNitroCooldownTicks} тиков с момента предыдущего ускорения.
}%end item
  \end{itemize}
}
\end{itemize}
}%end item
\divideents{setWheelTurn}
\item{\vskip -1.9ex 
\membername{setWheelTurn}
{\tt public void {\bf setWheelTurn}( {\tt double } {\bf wheelTurn} )
\label{l211}\label{l212}}%end signature
\begin{itemize}
\sld
\item{
\sld
{\bf Usage}
  \begin{itemize}\isep
   \item{
Задаёт относительный угол поворота колёс (или руля, что эквивалентно) кодемобиля.
 
% Removed by DocsPostProcessor:
% \textless p$/$\textgreater
% \\ Removed by DocsPostProcessor.
 
 Относительный угол должен лежать в интервале от {\tt -1.0} до {\tt 1.0}.
 Значения, выходящие за указанный интервал, будут приведены к ближайшей его границе.
 
% Removed by DocsPostProcessor:
% \textless p$/$\textgreater
% \\ Removed by DocsPostProcessor.
 
 Реальный поворот колёс может отличаться от установки, так как его изменение происходит не мгновенно, а
 со скоростью не более {\tt game.carWheelTurnChangePerTick} за тик. Поворот колёс создаёт добавочную угловую
 скорость кодемобиля (помимо угловой скорости, вызванной соударениями объектов и другими причинами), значение
 которой прямо пропорционально текущему относительному углу поворота колёс кодемобиля ({\tt car.wheelTurn}),
 коэффициенту {\tt game.carAngularSpeedFactor}, а также скалярному произведению вектора скорости кодемобиля и
 единичного вектора, направление которого совпадает с направлением кодемобиля.
}%end item
  \end{itemize}
}
\end{itemize}
}%end item
\end{itemize}
}
\hide{inherited}{
}
}
\startsection{Class}{OilSlick}{l8}{%
{\small Класс, определяющий лужу мазута. Содержит также все свойства круглого юнита.}
\vskip .1in 
\startsubsubsection{Declaration}{
\fbox{\vbox{
\hbox{\vbox{\small public 
class 
OilSlick}}
\noindent\hbox{\vbox{{\bf extends} CircularUnit}}
}}}

% Removed by DocsPostProcessor:
% \startsubsubsection{Constructors}{
% \vskip -2em
% \begin{itemize}
% \item{\vskip -1.9ex 
% \membername{OilSlick}
% {\tt public {\bf OilSlick}( {\tt long } {\bf id},
% {\tt double } {\bf mass},
% {\tt double } {\bf x},
% {\tt double } {\bf y},
% {\tt double } {\bf speedX},
% {\tt double } {\bf speedY},
% {\tt double } {\bf angle},
% {\tt double } {\bf angularSpeed},
% {\tt double } {\bf radius},
% {\tt int } {\bf remainingLifetime} )
% \label{l213}\label{l214}}%end signature
% }%end item
% \end{itemize}
% }
% \\ Removed by DocsPostProcessor.

\startsubsubsection{Methods}{
\vskip -2em
\begin{itemize}
\item{\vskip -1.9ex 
\membername{getRemainingLifetime}
{\tt public int {\bf getRemainingLifetime}(  )
\label{l215}\label{l216}}%end signature
\begin{itemize}
\sld
\item{{\bf Returns} - 
Возвращает количество тиков, по прошествии которого лужа мазута полностью высохнет. 
}%end item
\end{itemize}
}%end item
\end{itemize}
}
\hide{inherited}{
\startsubsubsection{Methods inherited from class {\tt CircularUnit}}{
\par{\small 
\refdefined{l4}\vskip -2em
\begin{itemize}
\item{\vskip -1.9ex 
\membername{getRadius}
{\tt public double {\bf getRadius}(  )
}%end signature
\begin{itemize}
\sld
\item{{\bf Returns} - 
Возвращает радиус объекта. 
}%end item
\end{itemize}
}%end item
\end{itemize}
}}
\startsubsubsection{Methods inherited from class {\tt Unit}}{
\par{\small 
\refdefined{l14}\vskip -2em
\begin{itemize}
\item{\vskip -1.9ex 
\membername{getAngle}
{\tt public final double {\bf getAngle}(  )
}%end signature
\begin{itemize}
\sld
\item{{\bf Returns} - 
Возвращает угол поворота объекта в радианах. Нулевой угол соответствует направлению оси абсцисс.
 Положительные значения соответствуют повороту по часовой стрелке. 
}%end item
\end{itemize}
}%end item
\divideents{getAngleTo}
\item{\vskip -1.9ex 
\membername{getAngleTo}
{\tt public double {\bf getAngleTo}( {\tt double } {\bf x},
{\tt double } {\bf y} )
}%end signature
\begin{itemize}
\sld
\item{
\sld
{\bf Parameters}
\sld\isep
  \begin{itemize}
\sld\isep
   \item{
\sld
{\tt x} - X-координата точки.}
   \item{
\sld
{\tt y} - Y-координата точки.}
  \end{itemize}
}%end item
\item{{\bf Returns} - 
Возвращает ориентированный угол [{\tt -PI}, {\tt PI}] между направлением
 данного объекта и вектором из центра данного объекта к указанной точке. 
}%end item
\end{itemize}
}%end item
\divideents{getAngleTo}
\item{\vskip -1.9ex 
\membername{getAngleTo}
{\tt public double {\bf getAngleTo}( {\tt Unit } {\bf unit} )
}%end signature
\begin{itemize}
\sld
\item{
\sld
{\bf Parameters}
\sld\isep
  \begin{itemize}
\sld\isep
   \item{
\sld
{\tt unit} - Объект, к центру которого необходимо определить угол.}
  \end{itemize}
}%end item
\item{{\bf Returns} - 
Возвращает ориентированный угол [{\tt -PI}, {\tt PI}] между направлением
 данного объекта и вектором из центра данного объекта к центру указанного объекта. 
}%end item
\end{itemize}
}%end item
\divideents{getAngularSpeed}
\item{\vskip -1.9ex 
\membername{getAngularSpeed}
{\tt public double {\bf getAngularSpeed}(  )
}%end signature
\begin{itemize}
\sld
\item{{\bf Returns} - 
Возвращает скорость вращения объекта.
 Положительные значения соответствуют вращению по часовой стрелке. 
}%end item
\end{itemize}
}%end item
\divideents{getDistanceTo}
\item{\vskip -1.9ex 
\membername{getDistanceTo}
{\tt public double {\bf getDistanceTo}( {\tt double } {\bf x},
{\tt double } {\bf y} )
}%end signature
\begin{itemize}
\sld
\item{
\sld
{\bf Parameters}
\sld\isep
  \begin{itemize}
\sld\isep
   \item{
\sld
{\tt x} - X-координата точки.}
   \item{
\sld
{\tt y} - Y-координата точки.}
  \end{itemize}
}%end item
\item{{\bf Returns} - 
Возвращает расстояние до точки от центра данного объекта. 
}%end item
\end{itemize}
}%end item
\divideents{getDistanceTo}
\item{\vskip -1.9ex 
\membername{getDistanceTo}
{\tt public double {\bf getDistanceTo}( {\tt Unit } {\bf unit} )
}%end signature
\begin{itemize}
\sld
\item{
\sld
{\bf Parameters}
\sld\isep
  \begin{itemize}
\sld\isep
   \item{
\sld
{\tt unit} - Объект, до центра которого необходимо определить расстояние.}
  \end{itemize}
}%end item
\item{{\bf Returns} - 
Возвращает расстояние от центра данного объекта до центра указанного объекта. 
}%end item
\end{itemize}
}%end item
\divideents{getId}
\item{\vskip -1.9ex 
\membername{getId}
{\tt public long {\bf getId}(  )
}%end signature
\begin{itemize}
\sld
\item{{\bf Returns} - 
Возвращает уникальный идентификатор объекта. 
}%end item
\end{itemize}
}%end item
\divideents{getMass}
\item{\vskip -1.9ex 
\membername{getMass}
{\tt public double {\bf getMass}(  )
}%end signature
\begin{itemize}
\sld
\item{{\bf Returns} - 
Возвращает массу объекта в единицах массы. 
}%end item
\end{itemize}
}%end item
\divideents{getSpeedX}
\item{\vskip -1.9ex 
\membername{getSpeedX}
{\tt public final double {\bf getSpeedX}(  )
}%end signature
\begin{itemize}
\sld
\item{{\bf Returns} - 
Возвращает X-составляющую скорости объекта. Ось абсцисс направлена слева направо. 
}%end item
\end{itemize}
}%end item
\divideents{getSpeedY}
\item{\vskip -1.9ex 
\membername{getSpeedY}
{\tt public final double {\bf getSpeedY}(  )
}%end signature
\begin{itemize}
\sld
\item{{\bf Returns} - 
Возвращает Y-составляющую скорости объекта. Ось ординат направлена сверху вниз. 
}%end item
\end{itemize}
}%end item
\divideents{getX}
\item{\vskip -1.9ex 
\membername{getX}
{\tt public final double {\bf getX}(  )
}%end signature
\begin{itemize}
\sld
\item{{\bf Returns} - 
Возвращает X-координату центра объекта. Ось абсцисс направлена слева направо. 
}%end item
\end{itemize}
}%end item
\divideents{getY}
\item{\vskip -1.9ex 
\membername{getY}
{\tt public final double {\bf getY}(  )
}%end signature
\begin{itemize}
\sld
\item{{\bf Returns} - 
Возвращает Y-координату центра объекта. Ось ординат направлена сверху вниз. 
}%end item
\end{itemize}
}%end item
\end{itemize}
}}
}
}
\startsection{Class}{Player}{l9}{%
{\small Содержит данные о текущем состоянии игрока.}
\vskip .1in 
\startsubsubsection{Declaration}{
\fbox{\vbox{
\hbox{\vbox{\small public 
class 
Player}}
\noindent\hbox{\vbox{{\bf extends} Object}}
}}}

% Removed by DocsPostProcessor:
% \startsubsubsection{Constructors}{
% \vskip -2em
% \begin{itemize}
% \item{\vskip -1.9ex 
% \membername{Player}
% {\tt public {\bf Player}( {\tt long } {\bf id},
% {\tt boolean } {\bf me},
% {\tt String } {\bf name},
% {\tt boolean } {\bf strategyCrashed},
% {\tt int } {\bf score} )
% \label{l217}\label{l218}}%end signature
% }%end item
% \end{itemize}
% }
% \\ Removed by DocsPostProcessor.

\startsubsubsection{Methods}{
\vskip -2em
\begin{itemize}
\item{\vskip -1.9ex 
\membername{getId}
{\tt public long {\bf getId}(  )
\label{l219}\label{l220}}%end signature
\begin{itemize}
\sld
\item{{\bf Returns} - 
Возвращает уникальный идентификатор игрока. 
}%end item
\end{itemize}
}%end item
\divideents{getName}
\item{\vskip -1.9ex 
\membername{getName}
{\tt public String {\bf getName}(  )
\label{l221}\label{l222}}%end signature
\begin{itemize}
\sld
\item{{\bf Returns} - 
Возвращает имя игрока. 
}%end item
\end{itemize}
}%end item
\divideents{getScore}
\item{\vskip -1.9ex 
\membername{getScore}
{\tt public int {\bf getScore}(  )
\label{l223}\label{l224}}%end signature
\begin{itemize}
\sld
\item{{\bf Returns} - 
Возвращает количество баллов, набранное игроком. 
}%end item
\end{itemize}
}%end item
\divideents{isMe}
\item{\vskip -1.9ex 
\membername{isMe}
{\tt public boolean {\bf isMe}(  )
\label{l225}\label{l226}}%end signature
\begin{itemize}
\sld
\item{{\bf Returns} - 
Возвращает {\tt true} в том и только в том случае, если этот игрок ваш. 
}%end item
\end{itemize}
}%end item
\divideents{isStrategyCrashed}
\item{\vskip -1.9ex 
\membername{isStrategyCrashed}
{\tt public boolean {\bf isStrategyCrashed}(  )
\label{l227}\label{l228}}%end signature
\begin{itemize}
\sld
\item{{\bf Returns} - 
Возвращает специальный флаг --- показатель того, что стратегия игрока \textless \textless упала\textgreater \textgreater .
 Более подробную информацию можно найти в документации к игре. 
}%end item
\end{itemize}
}%end item
\end{itemize}
}
\hide{inherited}{
}
}
\startsection{Class}{Projectile}{l10}{%
{\small Класс, определяющий метательный снаряд. Содержит также все свойства круглого юнита.}
\vskip .1in 
\startsubsubsection{Declaration}{
\fbox{\vbox{
\hbox{\vbox{\small public 
class 
Projectile}}
\noindent\hbox{\vbox{{\bf extends} CircularUnit}}
}}}

% Removed by DocsPostProcessor:
% \startsubsubsection{Constructors}{
% \vskip -2em
% \begin{itemize}
% \item{\vskip -1.9ex 
% \membername{Projectile}
% {\tt public {\bf Projectile}( {\tt long } {\bf id},
% {\tt double } {\bf mass},
% {\tt double } {\bf x},
% {\tt double } {\bf y},
% {\tt double } {\bf speedX},
% {\tt double } {\bf speedY},
% {\tt double } {\bf angle},
% {\tt double } {\bf angularSpeed},
% {\tt double } {\bf radius},
% {\tt long } {\bf carId},
% {\tt long } {\bf playerId},
% {\tt ProjectileType } {\bf type} )
% \label{l229}\label{l230}}%end signature
% }%end item
% \end{itemize}
% }
% \\ Removed by DocsPostProcessor.

\startsubsubsection{Methods}{
\vskip -2em
\begin{itemize}
\item{\vskip -1.9ex 
\membername{getCarId}
{\tt public long {\bf getCarId}(  )
\label{l231}\label{l232}}%end signature
\begin{itemize}
\sld
\item{{\bf Returns} - 
Возвращает идентификатор кодемобиля, выпустившего данный снаряд. 
}%end item
\end{itemize}
}%end item
\divideents{getPlayerId}
\item{\vskip -1.9ex 
\membername{getPlayerId}
{\tt public long {\bf getPlayerId}(  )
\label{l233}\label{l234}}%end signature
\begin{itemize}
\sld
\item{{\bf Returns} - 
Возвращает идентификатор игрока, кодемобиль которого выпустил данный снаряд. 
}%end item
\end{itemize}
}%end item
\divideents{getType}
\item{\vskip -1.9ex 
\membername{getType}
{\tt public ProjectileType {\bf getType}(  )
\label{l235}\label{l236}}%end signature
\begin{itemize}
\sld
\item{{\bf Returns} - 
Возвращает тип метательного снаряда. 
}%end item
\end{itemize}
}%end item
\end{itemize}
}
\hide{inherited}{
\startsubsubsection{Methods inherited from class {\tt CircularUnit}}{
\par{\small 
\refdefined{l4}\vskip -2em
\begin{itemize}
\item{\vskip -1.9ex 
\membername{getRadius}
{\tt public double {\bf getRadius}(  )
}%end signature
\begin{itemize}
\sld
\item{{\bf Returns} - 
Возвращает радиус объекта. 
}%end item
\end{itemize}
}%end item
\end{itemize}
}}
\startsubsubsection{Methods inherited from class {\tt Unit}}{
\par{\small 
\refdefined{l14}\vskip -2em
\begin{itemize}
\item{\vskip -1.9ex 
\membername{getAngle}
{\tt public final double {\bf getAngle}(  )
}%end signature
\begin{itemize}
\sld
\item{{\bf Returns} - 
Возвращает угол поворота объекта в радианах. Нулевой угол соответствует направлению оси абсцисс.
 Положительные значения соответствуют повороту по часовой стрелке. 
}%end item
\end{itemize}
}%end item
\divideents{getAngleTo}
\item{\vskip -1.9ex 
\membername{getAngleTo}
{\tt public double {\bf getAngleTo}( {\tt double } {\bf x},
{\tt double } {\bf y} )
}%end signature
\begin{itemize}
\sld
\item{
\sld
{\bf Parameters}
\sld\isep
  \begin{itemize}
\sld\isep
   \item{
\sld
{\tt x} - X-координата точки.}
   \item{
\sld
{\tt y} - Y-координата точки.}
  \end{itemize}
}%end item
\item{{\bf Returns} - 
Возвращает ориентированный угол [{\tt -PI}, {\tt PI}] между направлением
 данного объекта и вектором из центра данного объекта к указанной точке. 
}%end item
\end{itemize}
}%end item
\divideents{getAngleTo}
\item{\vskip -1.9ex 
\membername{getAngleTo}
{\tt public double {\bf getAngleTo}( {\tt Unit } {\bf unit} )
}%end signature
\begin{itemize}
\sld
\item{
\sld
{\bf Parameters}
\sld\isep
  \begin{itemize}
\sld\isep
   \item{
\sld
{\tt unit} - Объект, к центру которого необходимо определить угол.}
  \end{itemize}
}%end item
\item{{\bf Returns} - 
Возвращает ориентированный угол [{\tt -PI}, {\tt PI}] между направлением
 данного объекта и вектором из центра данного объекта к центру указанного объекта. 
}%end item
\end{itemize}
}%end item
\divideents{getAngularSpeed}
\item{\vskip -1.9ex 
\membername{getAngularSpeed}
{\tt public double {\bf getAngularSpeed}(  )
}%end signature
\begin{itemize}
\sld
\item{{\bf Returns} - 
Возвращает скорость вращения объекта.
 Положительные значения соответствуют вращению по часовой стрелке. 
}%end item
\end{itemize}
}%end item
\divideents{getDistanceTo}
\item{\vskip -1.9ex 
\membername{getDistanceTo}
{\tt public double {\bf getDistanceTo}( {\tt double } {\bf x},
{\tt double } {\bf y} )
}%end signature
\begin{itemize}
\sld
\item{
\sld
{\bf Parameters}
\sld\isep
  \begin{itemize}
\sld\isep
   \item{
\sld
{\tt x} - X-координата точки.}
   \item{
\sld
{\tt y} - Y-координата точки.}
  \end{itemize}
}%end item
\item{{\bf Returns} - 
Возвращает расстояние до точки от центра данного объекта. 
}%end item
\end{itemize}
}%end item
\divideents{getDistanceTo}
\item{\vskip -1.9ex 
\membername{getDistanceTo}
{\tt public double {\bf getDistanceTo}( {\tt Unit } {\bf unit} )
}%end signature
\begin{itemize}
\sld
\item{
\sld
{\bf Parameters}
\sld\isep
  \begin{itemize}
\sld\isep
   \item{
\sld
{\tt unit} - Объект, до центра которого необходимо определить расстояние.}
  \end{itemize}
}%end item
\item{{\bf Returns} - 
Возвращает расстояние от центра данного объекта до центра указанного объекта. 
}%end item
\end{itemize}
}%end item
\divideents{getId}
\item{\vskip -1.9ex 
\membername{getId}
{\tt public long {\bf getId}(  )
}%end signature
\begin{itemize}
\sld
\item{{\bf Returns} - 
Возвращает уникальный идентификатор объекта. 
}%end item
\end{itemize}
}%end item
\divideents{getMass}
\item{\vskip -1.9ex 
\membername{getMass}
{\tt public double {\bf getMass}(  )
}%end signature
\begin{itemize}
\sld
\item{{\bf Returns} - 
Возвращает массу объекта в единицах массы. 
}%end item
\end{itemize}
}%end item
\divideents{getSpeedX}
\item{\vskip -1.9ex 
\membername{getSpeedX}
{\tt public final double {\bf getSpeedX}(  )
}%end signature
\begin{itemize}
\sld
\item{{\bf Returns} - 
Возвращает X-составляющую скорости объекта. Ось абсцисс направлена слева направо. 
}%end item
\end{itemize}
}%end item
\divideents{getSpeedY}
\item{\vskip -1.9ex 
\membername{getSpeedY}
{\tt public final double {\bf getSpeedY}(  )
}%end signature
\begin{itemize}
\sld
\item{{\bf Returns} - 
Возвращает Y-составляющую скорости объекта. Ось ординат направлена сверху вниз. 
}%end item
\end{itemize}
}%end item
\divideents{getX}
\item{\vskip -1.9ex 
\membername{getX}
{\tt public final double {\bf getX}(  )
}%end signature
\begin{itemize}
\sld
\item{{\bf Returns} - 
Возвращает X-координату центра объекта. Ось абсцисс направлена слева направо. 
}%end item
\end{itemize}
}%end item
\divideents{getY}
\item{\vskip -1.9ex 
\membername{getY}
{\tt public final double {\bf getY}(  )
}%end signature
\begin{itemize}
\sld
\item{{\bf Returns} - 
Возвращает Y-координату центра объекта. Ось ординат направлена сверху вниз. 
}%end item
\end{itemize}
}%end item
\end{itemize}
}}
}
}
\startsection{Class}{ProjectileType}{l11}{%
{\small Тип метательного снаряда.}
\vskip .1in 
\startsubsubsection{Declaration}{
\fbox{\vbox{
\hbox{\vbox{\small public final 
class 
ProjectileType}}
\noindent\hbox{\vbox{{\bf extends} Enum}}
}}}
\startsubsubsection{Fields}{
\begin{itemize}
\item{
public static final ProjectileType WASHER\begin{itemize}\item{\vskip -.9ex Небольшой и лёгкий метательный снаряд для кодемобилей типа багги. Свободно пролетает сквозь ограждение трассы.
 Наносит относительно небольшой урон при попадании в кодемобиль и сразу после этого исчезает.}\end{itemize}
}
\item{
public static final ProjectileType TIRE\begin{itemize}\item{\vskip -.9ex Метательный снаряд, сопоставимый по размеру и массе с кодемобилем. Используется кодемобилями типа джип.
 Отражается как от кодемобилей, так и от границ трассы. Если скорость снаряда при столкновении меньше значения
 {\tt game.tireDisappearSpeedFactor * game.tireInitialSpeed}, то сразу после столкновения он исчезает.}\end{itemize}
}
\end{itemize}
}
\startsubsubsection{Methods}{
\vskip -2em
\begin{itemize}
\item{\vskip -1.9ex 
\membername{valueOf}
{\tt public static ProjectileType {\bf valueOf}( {\tt String } {\bf name} )
\label{l237}\label{l238}}%end signature
}%end item
\divideents{values}
\item{\vskip -1.9ex 
\membername{values}
{\tt public static ProjectileType[] {\bf values}(  )
\label{l239}\label{l240}}%end signature
}%end item
\end{itemize}
}
\hide{inherited}{
\startsubsubsection{Methods inherited from class {\tt Enum}}{
\par{\small 
\refdefined{l24}\vskip -2em
\begin{itemize}
\item{\vskip -1.9ex 
\membername{clone}
{\tt protected final Object {\bf clone}(  )
}%end signature
}%end item
\divideents{compareTo}
\item{\vskip -1.9ex 
\membername{compareTo}
{\tt public final int {\bf compareTo}( {\tt Enum } {\bf arg0} )
}%end signature
}%end item
\divideents{equals}
\item{\vskip -1.9ex 
\membername{equals}
{\tt public final boolean {\bf equals}( {\tt Object } {\bf arg0} )
}%end signature
}%end item
\divideents{finalize}
\item{\vskip -1.9ex 
\membername{finalize}
{\tt protected final void {\bf finalize}(  )
}%end signature
}%end item
\divideents{getDeclaringClass}
\item{\vskip -1.9ex 
\membername{getDeclaringClass}
{\tt public final Class {\bf getDeclaringClass}(  )
}%end signature
}%end item
\divideents{hashCode}
\item{\vskip -1.9ex 
\membername{hashCode}
{\tt public final int {\bf hashCode}(  )
}%end signature
}%end item
\divideents{name}
\item{\vskip -1.9ex 
\membername{name}
{\tt public final String {\bf name}(  )
}%end signature
}%end item
\divideents{ordinal}
\item{\vskip -1.9ex 
\membername{ordinal}
{\tt public final int {\bf ordinal}(  )
}%end signature
}%end item
\divideents{toString}
\item{\vskip -1.9ex 
\membername{toString}
{\tt public String {\bf toString}(  )
}%end signature
}%end item
\divideents{valueOf}
\item{\vskip -1.9ex 
\membername{valueOf}
{\tt public static Enum {\bf valueOf}( {\tt Class } {\bf arg0},
{\tt String } {\bf arg1} )
}%end signature
}%end item
\end{itemize}
}}
}
}
\startsection{Class}{RectangularUnit}{l12}{%
{\small Базовый класс для определения прямоугольных объектов. Содержит также все свойства юнита.}
\vskip .1in 
\startsubsubsection{Declaration}{
\fbox{\vbox{
\hbox{\vbox{\small public abstract 
class 
RectangularUnit}}
\noindent\hbox{\vbox{{\bf extends} Unit}}
}}}

% Removed by DocsPostProcessor:
% \startsubsubsection{Constructors}{
% \vskip -2em
% \begin{itemize}
% \item{\vskip -1.9ex 
% \membername{RectangularUnit}
% {\tt protected {\bf RectangularUnit}( {\tt long } {\bf id},
% {\tt double } {\bf mass},
% {\tt double } {\bf x},
% {\tt double } {\bf y},
% {\tt double } {\bf speedX},
% {\tt double } {\bf speedY},
% {\tt double } {\bf angle},
% {\tt double } {\bf angularSpeed},
% {\tt double } {\bf width},
% {\tt double } {\bf height} )
% \label{l241}\label{l242}}%end signature
% }%end item
% \end{itemize}
% }
% \\ Removed by DocsPostProcessor.

\startsubsubsection{Methods}{
\vskip -2em
\begin{itemize}
\item{\vskip -1.9ex 
\membername{getHeight}
{\tt public double {\bf getHeight}(  )
\label{l243}\label{l244}}%end signature
\begin{itemize}
\sld
\item{{\bf Returns} - 
Возвращает высоту объекта. 
}%end item
\end{itemize}
}%end item
\divideents{getWidth}
\item{\vskip -1.9ex 
\membername{getWidth}
{\tt public double {\bf getWidth}(  )
\label{l245}\label{l246}}%end signature
\begin{itemize}
\sld
\item{{\bf Returns} - 
Возвращает ширину объекта. 
}%end item
\end{itemize}
}%end item
\end{itemize}
}
\hide{inherited}{
\startsubsubsection{Methods inherited from class {\tt Unit}}{
\par{\small 
\refdefined{l14}\vskip -2em
\begin{itemize}
\item{\vskip -1.9ex 
\membername{getAngle}
{\tt public final double {\bf getAngle}(  )
}%end signature
\begin{itemize}
\sld
\item{{\bf Returns} - 
Возвращает угол поворота объекта в радианах. Нулевой угол соответствует направлению оси абсцисс.
 Положительные значения соответствуют повороту по часовой стрелке. 
}%end item
\end{itemize}
}%end item
\divideents{getAngleTo}
\item{\vskip -1.9ex 
\membername{getAngleTo}
{\tt public double {\bf getAngleTo}( {\tt double } {\bf x},
{\tt double } {\bf y} )
}%end signature
\begin{itemize}
\sld
\item{
\sld
{\bf Parameters}
\sld\isep
  \begin{itemize}
\sld\isep
   \item{
\sld
{\tt x} - X-координата точки.}
   \item{
\sld
{\tt y} - Y-координата точки.}
  \end{itemize}
}%end item
\item{{\bf Returns} - 
Возвращает ориентированный угол [{\tt -PI}, {\tt PI}] между направлением
 данного объекта и вектором из центра данного объекта к указанной точке. 
}%end item
\end{itemize}
}%end item
\divideents{getAngleTo}
\item{\vskip -1.9ex 
\membername{getAngleTo}
{\tt public double {\bf getAngleTo}( {\tt Unit } {\bf unit} )
}%end signature
\begin{itemize}
\sld
\item{
\sld
{\bf Parameters}
\sld\isep
  \begin{itemize}
\sld\isep
   \item{
\sld
{\tt unit} - Объект, к центру которого необходимо определить угол.}
  \end{itemize}
}%end item
\item{{\bf Returns} - 
Возвращает ориентированный угол [{\tt -PI}, {\tt PI}] между направлением
 данного объекта и вектором из центра данного объекта к центру указанного объекта. 
}%end item
\end{itemize}
}%end item
\divideents{getAngularSpeed}
\item{\vskip -1.9ex 
\membername{getAngularSpeed}
{\tt public double {\bf getAngularSpeed}(  )
}%end signature
\begin{itemize}
\sld
\item{{\bf Returns} - 
Возвращает скорость вращения объекта.
 Положительные значения соответствуют вращению по часовой стрелке. 
}%end item
\end{itemize}
}%end item
\divideents{getDistanceTo}
\item{\vskip -1.9ex 
\membername{getDistanceTo}
{\tt public double {\bf getDistanceTo}( {\tt double } {\bf x},
{\tt double } {\bf y} )
}%end signature
\begin{itemize}
\sld
\item{
\sld
{\bf Parameters}
\sld\isep
  \begin{itemize}
\sld\isep
   \item{
\sld
{\tt x} - X-координата точки.}
   \item{
\sld
{\tt y} - Y-координата точки.}
  \end{itemize}
}%end item
\item{{\bf Returns} - 
Возвращает расстояние до точки от центра данного объекта. 
}%end item
\end{itemize}
}%end item
\divideents{getDistanceTo}
\item{\vskip -1.9ex 
\membername{getDistanceTo}
{\tt public double {\bf getDistanceTo}( {\tt Unit } {\bf unit} )
}%end signature
\begin{itemize}
\sld
\item{
\sld
{\bf Parameters}
\sld\isep
  \begin{itemize}
\sld\isep
   \item{
\sld
{\tt unit} - Объект, до центра которого необходимо определить расстояние.}
  \end{itemize}
}%end item
\item{{\bf Returns} - 
Возвращает расстояние от центра данного объекта до центра указанного объекта. 
}%end item
\end{itemize}
}%end item
\divideents{getId}
\item{\vskip -1.9ex 
\membername{getId}
{\tt public long {\bf getId}(  )
}%end signature
\begin{itemize}
\sld
\item{{\bf Returns} - 
Возвращает уникальный идентификатор объекта. 
}%end item
\end{itemize}
}%end item
\divideents{getMass}
\item{\vskip -1.9ex 
\membername{getMass}
{\tt public double {\bf getMass}(  )
}%end signature
\begin{itemize}
\sld
\item{{\bf Returns} - 
Возвращает массу объекта в единицах массы. 
}%end item
\end{itemize}
}%end item
\divideents{getSpeedX}
\item{\vskip -1.9ex 
\membername{getSpeedX}
{\tt public final double {\bf getSpeedX}(  )
}%end signature
\begin{itemize}
\sld
\item{{\bf Returns} - 
Возвращает X-составляющую скорости объекта. Ось абсцисс направлена слева направо. 
}%end item
\end{itemize}
}%end item
\divideents{getSpeedY}
\item{\vskip -1.9ex 
\membername{getSpeedY}
{\tt public final double {\bf getSpeedY}(  )
}%end signature
\begin{itemize}
\sld
\item{{\bf Returns} - 
Возвращает Y-составляющую скорости объекта. Ось ординат направлена сверху вниз. 
}%end item
\end{itemize}
}%end item
\divideents{getX}
\item{\vskip -1.9ex 
\membername{getX}
{\tt public final double {\bf getX}(  )
}%end signature
\begin{itemize}
\sld
\item{{\bf Returns} - 
Возвращает X-координату центра объекта. Ось абсцисс направлена слева направо. 
}%end item
\end{itemize}
}%end item
\divideents{getY}
\item{\vskip -1.9ex 
\membername{getY}
{\tt public final double {\bf getY}(  )
}%end signature
\begin{itemize}
\sld
\item{{\bf Returns} - 
Возвращает Y-координату центра объекта. Ось ординат направлена сверху вниз. 
}%end item
\end{itemize}
}%end item
\end{itemize}
}}
}
}
\startsection{Class}{TileType}{l13}{%
{\small Тип тайла.}
\vskip .1in 
\startsubsubsection{Declaration}{
\fbox{\vbox{
\hbox{\vbox{\small public final 
class 
TileType}}
\noindent\hbox{\vbox{{\bf extends} Enum}}
}}}
\startsubsubsection{Fields}{
\begin{itemize}
\item{
public static final TileType EMPTY\begin{itemize}\item{\vskip -.9ex Пустой тайл.}\end{itemize}
}
\item{
public static final TileType VERTICAL\begin{itemize}\item{\vskip -.9ex Тайл с прямым вертикальным участком дороги.}\end{itemize}
}
\item{
public static final TileType HORIZONTAL\begin{itemize}\item{\vskip -.9ex Тайл с прямым горизонтальным участком дороги.}\end{itemize}
}
\item{
public static final TileType LEFT\_TOP\_CORNER\begin{itemize}\item{\vskip -.9ex Тайл, выполняющий роль сочленения двух других тайлов: справа и снизу от данного тайла.}\end{itemize}
}
\item{
public static final TileType RIGHT\_TOP\_CORNER\begin{itemize}\item{\vskip -.9ex Тайл, выполняющий роль сочленения двух других тайлов: слева и снизу от данного тайла.}\end{itemize}
}
\item{
public static final TileType LEFT\_BOTTOM\_CORNER\begin{itemize}\item{\vskip -.9ex Тайл, выполняющий роль сочленения двух других тайлов: справа и сверху от данного тайла.}\end{itemize}
}
\item{
public static final TileType RIGHT\_BOTTOM\_CORNER\begin{itemize}\item{\vskip -.9ex Тайл, выполняющий роль сочленения двух других тайлов: слева и сверху от данного тайла.}\end{itemize}
}
\item{
public static final TileType LEFT\_HEADED\_T\begin{itemize}\item{\vskip -.9ex Тайл, выполняющий роль сочленения трёх других тайлов: слева, снизу и сверху от данного тайла.}\end{itemize}
}
\item{
public static final TileType RIGHT\_HEADED\_T\begin{itemize}\item{\vskip -.9ex Тайл, выполняющий роль сочленения трёх других тайлов: справа, снизу и сверху от данного тайла.}\end{itemize}
}
\item{
public static final TileType TOP\_HEADED\_T\begin{itemize}\item{\vskip -.9ex Тайл, выполняющий роль сочленения трёх других тайлов: слева, справа и сверху от данного тайла.}\end{itemize}
}
\item{
public static final TileType BOTTOM\_HEADED\_T\begin{itemize}\item{\vskip -.9ex Тайл, выполняющий роль сочленения трёх других тайлов: слева, справа и снизу от данного тайла.}\end{itemize}
}
\item{
public static final TileType CROSSROADS\begin{itemize}\item{\vskip -.9ex Тайл, выполняющий роль сочленения четырёх других тайлов: со всех сторон от данного тайла.}\end{itemize}
}
\item{
public static final TileType UNKNOWN\begin{itemize}\item{\vskip -.9ex Тип тайла пока не известен.}\end{itemize}
}
\end{itemize}
}
\startsubsubsection{Methods}{
\vskip -2em
\begin{itemize}
\item{\vskip -1.9ex 
\membername{valueOf}
{\tt public static TileType {\bf valueOf}( {\tt String } {\bf name} )
\label{l247}\label{l248}}%end signature
}%end item
\divideents{values}
\item{\vskip -1.9ex 
\membername{values}
{\tt public static TileType[] {\bf values}(  )
\label{l249}\label{l250}}%end signature
}%end item
\end{itemize}
}
\hide{inherited}{
\startsubsubsection{Methods inherited from class {\tt Enum}}{
\par{\small 
\refdefined{l24}\vskip -2em
\begin{itemize}
\item{\vskip -1.9ex 
\membername{clone}
{\tt protected final Object {\bf clone}(  )
}%end signature
}%end item
\divideents{compareTo}
\item{\vskip -1.9ex 
\membername{compareTo}
{\tt public final int {\bf compareTo}( {\tt Enum } {\bf arg0} )
}%end signature
}%end item
\divideents{equals}
\item{\vskip -1.9ex 
\membername{equals}
{\tt public final boolean {\bf equals}( {\tt Object } {\bf arg0} )
}%end signature
}%end item
\divideents{finalize}
\item{\vskip -1.9ex 
\membername{finalize}
{\tt protected final void {\bf finalize}(  )
}%end signature
}%end item
\divideents{getDeclaringClass}
\item{\vskip -1.9ex 
\membername{getDeclaringClass}
{\tt public final Class {\bf getDeclaringClass}(  )
}%end signature
}%end item
\divideents{hashCode}
\item{\vskip -1.9ex 
\membername{hashCode}
{\tt public final int {\bf hashCode}(  )
}%end signature
}%end item
\divideents{name}
\item{\vskip -1.9ex 
\membername{name}
{\tt public final String {\bf name}(  )
}%end signature
}%end item
\divideents{ordinal}
\item{\vskip -1.9ex 
\membername{ordinal}
{\tt public final int {\bf ordinal}(  )
}%end signature
}%end item
\divideents{toString}
\item{\vskip -1.9ex 
\membername{toString}
{\tt public String {\bf toString}(  )
}%end signature
}%end item
\divideents{valueOf}
\item{\vskip -1.9ex 
\membername{valueOf}
{\tt public static Enum {\bf valueOf}( {\tt Class } {\bf arg0},
{\tt String } {\bf arg1} )
}%end signature
}%end item
\end{itemize}
}}
}
}
\startsection{Class}{Unit}{l14}{%
{\small Базовый класс для определения объектов (\textless \textless юнитов\textgreater \textgreater ) на игровом поле.}
\vskip .1in 
\startsubsubsection{Declaration}{
\fbox{\vbox{
\hbox{\vbox{\small public abstract 
class 
Unit}}
\noindent\hbox{\vbox{{\bf extends} Object}}
}}}

% Removed by DocsPostProcessor:
% \startsubsubsection{Constructors}{
% \vskip -2em
% \begin{itemize}
% \item{\vskip -1.9ex 
% \membername{Unit}
% {\tt protected {\bf Unit}( {\tt long } {\bf id},
% {\tt double } {\bf mass},
% {\tt double } {\bf x},
% {\tt double } {\bf y},
% {\tt double } {\bf speedX},
% {\tt double } {\bf speedY},
% {\tt double } {\bf angle},
% {\tt double } {\bf angularSpeed} )
% \label{l251}\label{l252}}%end signature
% }%end item
% \end{itemize}
% }
% \\ Removed by DocsPostProcessor.

\startsubsubsection{Methods}{
\vskip -2em
\begin{itemize}
\item{\vskip -1.9ex 
\membername{getAngle}
{\tt public final double {\bf getAngle}(  )
\label{l253}\label{l254}}%end signature
\begin{itemize}
\sld
\item{{\bf Returns} - 
Возвращает угол поворота объекта в радианах. Нулевой угол соответствует направлению оси абсцисс.
 Положительные значения соответствуют повороту по часовой стрелке. 
}%end item
\end{itemize}
}%end item
\divideents{getAngleTo}
\item{\vskip -1.9ex 
\membername{getAngleTo}
{\tt public double {\bf getAngleTo}( {\tt double } {\bf x},
{\tt double } {\bf y} )
\label{l255}\label{l256}}%end signature
\begin{itemize}
\sld
\item{
\sld
{\bf Parameters}
\sld\isep
  \begin{itemize}
\sld\isep
   \item{
\sld
{\tt x} - X-координата точки.}
   \item{
\sld
{\tt y} - Y-координата точки.}
  \end{itemize}
}%end item
\item{{\bf Returns} - 
Возвращает ориентированный угол [{\tt -PI}, {\tt PI}] между направлением
 данного объекта и вектором из центра данного объекта к указанной точке. 
}%end item
\end{itemize}
}%end item
\divideents{getAngleTo}
\item{\vskip -1.9ex 
\membername{getAngleTo}
{\tt public double {\bf getAngleTo}( {\tt Unit } {\bf unit} )
\label{l257}\label{l258}}%end signature
\begin{itemize}
\sld
\item{
\sld
{\bf Parameters}
\sld\isep
  \begin{itemize}
\sld\isep
   \item{
\sld
{\tt unit} - Объект, к центру которого необходимо определить угол.}
  \end{itemize}
}%end item
\item{{\bf Returns} - 
Возвращает ориентированный угол [{\tt -PI}, {\tt PI}] между направлением
 данного объекта и вектором из центра данного объекта к центру указанного объекта. 
}%end item
\end{itemize}
}%end item
\divideents{getAngularSpeed}
\item{\vskip -1.9ex 
\membername{getAngularSpeed}
{\tt public double {\bf getAngularSpeed}(  )
\label{l259}\label{l260}}%end signature
\begin{itemize}
\sld
\item{{\bf Returns} - 
Возвращает скорость вращения объекта.
 Положительные значения соответствуют вращению по часовой стрелке. 
}%end item
\end{itemize}
}%end item
\divideents{getDistanceTo}
\item{\vskip -1.9ex 
\membername{getDistanceTo}
{\tt public double {\bf getDistanceTo}( {\tt double } {\bf x},
{\tt double } {\bf y} )
\label{l261}\label{l262}}%end signature
\begin{itemize}
\sld
\item{
\sld
{\bf Parameters}
\sld\isep
  \begin{itemize}
\sld\isep
   \item{
\sld
{\tt x} - X-координата точки.}
   \item{
\sld
{\tt y} - Y-координата точки.}
  \end{itemize}
}%end item
\item{{\bf Returns} - 
Возвращает расстояние до точки от центра данного объекта. 
}%end item
\end{itemize}
}%end item
\divideents{getDistanceTo}
\item{\vskip -1.9ex 
\membername{getDistanceTo}
{\tt public double {\bf getDistanceTo}( {\tt Unit } {\bf unit} )
\label{l263}\label{l264}}%end signature
\begin{itemize}
\sld
\item{
\sld
{\bf Parameters}
\sld\isep
  \begin{itemize}
\sld\isep
   \item{
\sld
{\tt unit} - Объект, до центра которого необходимо определить расстояние.}
  \end{itemize}
}%end item
\item{{\bf Returns} - 
Возвращает расстояние от центра данного объекта до центра указанного объекта. 
}%end item
\end{itemize}
}%end item
\divideents{getId}
\item{\vskip -1.9ex 
\membername{getId}
{\tt public long {\bf getId}(  )
\label{l265}\label{l266}}%end signature
\begin{itemize}
\sld
\item{{\bf Returns} - 
Возвращает уникальный идентификатор объекта. 
}%end item
\end{itemize}
}%end item
\divideents{getMass}
\item{\vskip -1.9ex 
\membername{getMass}
{\tt public double {\bf getMass}(  )
\label{l267}\label{l268}}%end signature
\begin{itemize}
\sld
\item{{\bf Returns} - 
Возвращает массу объекта в единицах массы. 
}%end item
\end{itemize}
}%end item
\divideents{getSpeedX}
\item{\vskip -1.9ex 
\membername{getSpeedX}
{\tt public final double {\bf getSpeedX}(  )
\label{l269}\label{l270}}%end signature
\begin{itemize}
\sld
\item{{\bf Returns} - 
Возвращает X-составляющую скорости объекта. Ось абсцисс направлена слева направо. 
}%end item
\end{itemize}
}%end item
\divideents{getSpeedY}
\item{\vskip -1.9ex 
\membername{getSpeedY}
{\tt public final double {\bf getSpeedY}(  )
\label{l271}\label{l272}}%end signature
\begin{itemize}
\sld
\item{{\bf Returns} - 
Возвращает Y-составляющую скорости объекта. Ось ординат направлена сверху вниз. 
}%end item
\end{itemize}
}%end item
\divideents{getX}
\item{\vskip -1.9ex 
\membername{getX}
{\tt public final double {\bf getX}(  )
\label{l273}\label{l274}}%end signature
\begin{itemize}
\sld
\item{{\bf Returns} - 
Возвращает X-координату центра объекта. Ось абсцисс направлена слева направо. 
}%end item
\end{itemize}
}%end item
\divideents{getY}
\item{\vskip -1.9ex 
\membername{getY}
{\tt public final double {\bf getY}(  )
\label{l275}\label{l276}}%end signature
\begin{itemize}
\sld
\item{{\bf Returns} - 
Возвращает Y-координату центра объекта. Ось ординат направлена сверху вниз. 
}%end item
\end{itemize}
}%end item
\end{itemize}
}
\hide{inherited}{
}
}
\startsection{Class}{World}{l15}{%
{\small Этот класс описывает игровой мир. Содержит также описания всех игроков и игровых объектов (\textless \textless юнитов\textgreater \textgreater ).}
\vskip .1in 
\startsubsubsection{Declaration}{
\fbox{\vbox{
\hbox{\vbox{\small public 
class 
World}}
\noindent\hbox{\vbox{{\bf extends} Object}}
}}}

% Removed by DocsPostProcessor:
% \startsubsubsection{Constructors}{
% \vskip -2em
% \begin{itemize}
% \item{\vskip -1.9ex 
% \membername{World}
% {\tt public {\bf World}( {\tt int } {\bf tick},
% {\tt int } {\bf tickCount},
% {\tt int } {\bf lastTickIndex},
% {\tt int } {\bf width},
% {\tt int } {\bf height},
% {\tt Player[]} {\bf players},
% {\tt Car[]} {\bf cars},
% {\tt Projectile[]} {\bf projectiles},
% {\tt Bonus[]} {\bf bonuses},
% {\tt OilSlick[]} {\bf oilSlicks},
% {\tt String } {\bf mapName},
% {\tt TileType[][]} {\bf tilesXY},
% {\tt int[][]} {\bf waypoints},
% {\tt Direction } {\bf startingDirection} )
% \label{l277}\label{l278}}%end signature
% }%end item
% \end{itemize}
% }
% \\ Removed by DocsPostProcessor.

\startsubsubsection{Methods}{
\vskip -2em
\begin{itemize}
\item{\vskip -1.9ex 
\membername{getBonuses}
{\tt public Bonus[] {\bf getBonuses}(  )
\label{l279}\label{l280}}%end signature
\begin{itemize}
\sld
\item{{\bf Returns} - 
Возвращает список бонусов (в случайном порядке).
 После каждого тика объекты, задающие бонусы, пересоздаются. 
}%end item
\end{itemize}
}%end item
\divideents{getCars}
\item{\vskip -1.9ex 
\membername{getCars}
{\tt public Car[] {\bf getCars}(  )
\label{l281}\label{l282}}%end signature
\begin{itemize}
\sld
\item{{\bf Returns} - 
Возвращает список кодемобилей (в случайном порядке).
 После каждого тика объекты, задающие кодемобили, пересоздаются. 
}%end item
\end{itemize}
}%end item
\divideents{getHeight}
\item{\vskip -1.9ex 
\membername{getHeight}
{\tt public int {\bf getHeight}(  )
\label{l283}\label{l284}}%end signature
\begin{itemize}
\sld
\item{{\bf Returns} - 
Возвращает высоту мира в тайлах. 
}%end item
\end{itemize}
}%end item
\divideents{getLastTickIndex}
\item{\vskip -1.9ex 
\membername{getLastTickIndex}
{\tt public int {\bf getLastTickIndex}(  )
\label{l285}\label{l286}}%end signature
\begin{itemize}
\sld
\item{{\bf Returns} - 
Возвращает номер последнего тика игры.
 Сразу после старта содержит значение {\tt tickCount - 1}.
 В момент завершения трассы любым из кодемобилей получает значение
 {\tt min(tickCount - 1, tick + max(floor(game.burningTimeDurationFactor * game.lapTickCount), 1))}.
 Таким образом, кодемобиль, отставший от идущего впереди более, чем на {\tt game.burningTimeDurationFactor}
 кругов, рискует не успеть вообще добраться до финиша.
 
% Removed by DocsPostProcessor:
% \textless p$/$\textgreater
% \\ Removed by DocsPostProcessor.
 
 Игра может закончиться раньше, чем наступит {\tt lastTickIndex}, если для каждого игрока выполняется одно
 из двух условий: стратегия игрока \textless \textless упала\textgreater \textgreater , либо все его кодемобили финишировали. 
}%end item
\end{itemize}
}%end item
\divideents{getMapName}
\item{\vskip -1.9ex 
\membername{getMapName}
{\tt public String {\bf getMapName}(  )
\label{l287}\label{l288}}%end signature
\begin{itemize}
\sld
\item{{\bf Returns} - 
Возвращает краткое уникальное название трассы. 
}%end item
\end{itemize}
}%end item
\divideents{getMyPlayer}
\item{\vskip -1.9ex 
\membername{getMyPlayer}
{\tt public Player {\bf getMyPlayer}(  )
\label{l289}\label{l290}}%end signature
\begin{itemize}
\sld
\item{{\bf Returns} - 
Возвращает вашего игрока. 
}%end item
\end{itemize}
}%end item
\divideents{getOilSlicks}
\item{\vskip -1.9ex 
\membername{getOilSlicks}
{\tt public OilSlick[] {\bf getOilSlicks}(  )
\label{l291}\label{l292}}%end signature
\begin{itemize}
\sld
\item{{\bf Returns} - 
Возвращает список масляных луж (в случайном порядке).
 После каждого тика объекты, задающие лужи, пересоздаются. 
}%end item
\end{itemize}
}%end item
\divideents{getPlayers}
\item{\vskip -1.9ex 
\membername{getPlayers}
{\tt public Player[] {\bf getPlayers}(  )
\label{l293}\label{l294}}%end signature
\begin{itemize}
\sld
\item{{\bf Returns} - 
Возвращает список игроков (в случайном порядке).
 После каждого тика объекты, задающие игроков, пересоздаются. 
}%end item
\end{itemize}
}%end item
\divideents{getProjectiles}
\item{\vskip -1.9ex 
\membername{getProjectiles}
{\tt public Projectile[] {\bf getProjectiles}(  )
\label{l295}\label{l296}}%end signature
\begin{itemize}
\sld
\item{{\bf Returns} - 
Возвращает список снарядов (в случайном порядке).
 После каждого тика объекты, задающие снаряды, пересоздаются. 
}%end item
\end{itemize}
}%end item
\divideents{getStartingDirection}
\item{\vskip -1.9ex 
\membername{getStartingDirection}
{\tt public Direction {\bf getStartingDirection}(  )
\label{l297}\label{l298}}%end signature
\begin{itemize}
\sld
\item{{\bf Returns} - 
Направление кодемобиля в начале игры. 
}%end item
\end{itemize}
}%end item
\divideents{getTick}
\item{\vskip -1.9ex 
\membername{getTick}
{\tt public int {\bf getTick}(  )
\label{l299}\label{l300}}%end signature
\begin{itemize}
\sld
\item{{\bf Returns} - 
Возвращает номер текущего тика. 
}%end item
\end{itemize}
}%end item
\divideents{getTickCount}
\item{\vskip -1.9ex 
\membername{getTickCount}
{\tt public int {\bf getTickCount}(  )
\label{l301}\label{l302}}%end signature
\begin{itemize}
\sld
\item{{\bf Returns} - 
Возвращает базовую длительность игры в тиках.
 Реальная длительность может отличаться от этого значения в меньшую сторону.
 Поле может быть определено как {\tt game.initialFreezeDurationTicks + game.lapCount * game.lapTickCount}.
 Значение поля не меняется в процессе игры. Эквивалентно {\tt game.tickCount}. 
}%end item
\end{itemize}
}%end item
\divideents{getTilesXY}
\item{\vskip -1.9ex 
\membername{getTilesXY}
{\tt public TileType[][] {\bf getTilesXY}(  )
\label{l303}\label{l304}}%end signature
\begin{itemize}
\sld
\item{{\bf Returns} - 
Возвращает двумерный массив тайлов, где первое измерение --- это позиция X, а второе --- Y.
 Конвертировать позицию в точные координаты можно, используя значение {\tt game.trackTileSize}. 
}%end item
\end{itemize}
}%end item
\divideents{getWaypoints}
\item{\vskip -1.9ex 
\membername{getWaypoints}
{\tt public int[][] {\bf getWaypoints}(  )
\label{l305}\label{l306}}%end signature
\begin{itemize}
\sld
\item{{\bf Returns} - 
Возвращает массив ключевых тайлов. Каждый тайл задаётся массивом длины 2,
 где элемент с индексом {\tt 0} содержит позицию X, а элемент с индексом {\tt 1} --- позицию Y.
 Конвертировать позицию в точные координаты можно, используя значение {\tt game.trackTileSize}.
 Для прохождения круга кодемобилю необходимо посещать тайлы в указанном порядке.
 Ключевой тайл с индексом {\tt 0} является одновременно начальным тайлом трассы и конечным тайлом каждого круга.
 Считается, что кодемобиль посетил ключевой тайл, если центр кодемобиля пересёк границу этого тайла. 
}%end item
\end{itemize}
}%end item
\divideents{getWidth}
\item{\vskip -1.9ex 
\membername{getWidth}
{\tt public int {\bf getWidth}(  )
\label{l307}\label{l308}}%end signature
\begin{itemize}
\sld
\item{{\bf Returns} - 
Возвращает ширину мира в тайлах. 
}%end item
\end{itemize}
}%end item
\end{itemize}
}
\hide{inherited}{
}
}
}
}
\newpage
\def\packagename{\textless none\textgreater }
\chapter{\bf Package \textless none\textgreater }{
\vskip -.25in
\hbox to \hsize{\it Package Contents\hfil Page}
\rule{\hsize}{.7mm}
\vskip .13in
\hbox{\bf Interfaces}
\entityintro{Strategy}{l309}{Стратегия --- интерфейс, содержащий описание методов искусственного интеллекта кодемобиля.}
\vskip .1in
\rule{\hsize}{.7mm}
\vskip .1in
\newpage
\section{Interfaces}{
\startsection{Interface}{Strategy}{l309}{%
{\small Стратегия --- интерфейс, содержащий описание методов искусственного интеллекта кодемобиля.
 Каждая пользовательская стратегия должна реализовывать этот интерфейс.
 Может отсутствовать в некоторых языковых пакетах, если язык не поддерживает интерфейсы.}
\vskip .1in 
\startsubsubsection{Declaration}{
\fbox{\vbox{
\hbox{\vbox{\small public interface 
Strategy}}
}}}
\startsubsubsection{Methods}{
\vskip -2em
\begin{itemize}
\item{\vskip -1.9ex 
\membername{move}
{\tt public void {\bf move}( {\tt Car } {\bf self},
{\tt World } {\bf world},
{\tt Game } {\bf game},
{\tt Move } {\bf move} )
\label{l310}\label{l311}}%end signature
\begin{itemize}
\sld
\item{
\sld
{\bf Usage}
  \begin{itemize}\isep
   \item{
Основной метод стратегии, осуществляющий управление кодемобилем.
 Вызывается каждый тик для каждого кодемобиля.
}%end item
  \end{itemize}
}
\item{
\sld
{\bf Parameters}
\sld\isep
  \begin{itemize}
\sld\isep
   \item{
\sld
{\tt self} - Кодемобиль, которым данный метод будет осуществлять управление.}
   \item{
\sld
{\tt world} - Текущее состояние мира.}
   \item{
\sld
{\tt game} - Различные игровые константы.}
   \item{
\sld
{\tt move} - Результатом работы метода является изменение полей данного объекта.}
  \end{itemize}
}%end item
\end{itemize}
}%end item
\end{itemize}
}
\hide{inherited}{
}
}
}
}
\end{document}
